\section{反馈线性化}\label{5Bref}
\textbf{基本思想:}若能将非线性的动力学转化为线性的,那么便可采用控制线性系统的手段来控制非线性系统。
\begin{note}
    这和通过(利用雅可比矩阵的)局部线性化完全不同,因为反馈线性化是通过{\bf 精确的}状态变换和反馈全局完成的,而不是通过取动力学的局部线性{\bf 近似}。
\end{note}

{\bf 问题陈述:} 给定$\dot{x} = f (x, u), x \in \mathbb{R}^n, u
  \in \mathbb{R}^m$,求出函数$g, h$、矩阵$A \in \mathbb{R}^{n \times n}, B \in \mathbb{R}^{n\times m}$,以及新变量$z \in
  \mathbb{R}^n, v \in \mathbb{R}^m$,使得
  \begin{align*}
      u &= g (x, v)&\text{{\small(控制输入,将系统本来的控制输入和线性化系统的控制输入联系起来)}}\\
      z &= h (x),&\text{{\small(原状态与新状态的变换,将原状态与线性化后的新状态联系起来)}}\\
      \dot{z} &= A  z + B  v,&\text{(新状态满足的线性方程)}
  \end{align*}
然后利用控制线性系统的方法来设计$v$。

\subsection{输入到状态线性化}
\begin{example}[可控标准型]
考虑下述系统
    \begin{align*}
    \dot{x}_1 & =  x_2\\
     \dot{x}_2 & =  x_3\\
     \vdots &   \\
     \dot{x}_{n - 1} & = x_n\\
    \dot{x}_n & = f (x) + b (x) u
\end{align*}
假设$b \neq 0$,设计如下控制输入
\[ u = \frac{1}{b (x)} (v - f (x)) \]
则我们可消去非线性项,而得到
$ \dot{x}_n = v $,
这就将系统转化为可控标准型。控制律可选择为
\[ v = - k_0 x_1 - k_1 x_2 - \cdots - k_{n - 1} x_n \]
其中各 $k_i$ 应使得多项式(其实就是线性化后系统的特征多项式)
\[ p^n + k_{n - 1} p^{n - 1} + \cdots + k_0 = 0 \]
的所有根都具有负实部,这样就可使 $x (t) \rightarrow 0$。
\end{example}
\begin{example}[更一般的例子]
  考虑下述系统
  \begin{align*}
    \dot{x}_1 & = - 2 x_1 + a  x_2 + \sin  x_1\\
    \dot{x}_2 & = - x_2 \cos  x_1 + u  \cos (2 x_2)
  \end{align*}
  本例的难点主要在于,第一个方程中的非线性项无法直接被$u$消去。于是,我们寻求状态变换$z = z (x)$与输入变换$u = u (x, v)$使得非线性系统转为线性的。
  
  选取状态变换 $z_1 = x_1, z_2 = a  x_2 + \sin  x_1$,那么 $a  x_2 = z_2 -\sin  z_1$。因此
  \begin{align*}
    \dot{z}_1 & = - 2 z_1 + z_2\\
    \dot{z}_2 & = a  \dot{x}_2 + \cos  x_1 \cdot \dot{x}_1\\
    & = a  (- x_2 \cos  x_1 + u  \cos (2 x_2)) + \cos  x_1 (- 2 x_1 + a 
    x_2 + \sin  x_1)\\
    & = (\sin  x_1 - 2 x_1) \cos  x_1 + u  a   \cos (2 x_2)\\
    & \triangleq f(x)+g(x)u
  \end{align*}
  选取 $u = \frac{1}{g(x)} (v - f(x))$,我们就得到下述线性系统
  \[ \left\{\begin{array}{l}
       \dot{z}_1 = - 2 z_1 + z_2\\
       \dot{z}_2 = v
     \end{array}\right. \]
  该系统是能控的。那么,采用控制律$v = - k_1 z_1 - k_2 z_2$,只要增益$k_1$ 和 $k_2$设置得当,我们就能将极点配置在左半平面任意我们需要的位置上,而镇定这个系统。
  
  \begin{remark}   
    \begin{enumerate}[leftmargin=2em]
      \item 上面获得的结果并非全局的,因为当$g(x)=0$时$u$没有定义。
      
      \item 为了实际应用上述控制律,新状态变量$(z_1,z_2)$必定要能获取。也即,系统的全部状态$(x_1, x_2)$都要可测,才能计算出所需的$u$。
    \end{enumerate}
  \end{remark}
\end{example}

\subsection{输入到输出线性化}

在原系统上增加输出 $y$:
\[ \left\{\begin{array}{l}
     \dot{x} = f (x, u)\\
     y = h (x)
   \end{array}\right. \]
目标是:设计 $u$ 使得$y \rightarrow 0$。我们希望得出 $y$ 与 $u$之间简单而直接的表达式。

\begin{example}
    考虑下述三阶系统
    \begin{align*}
  \dot{x}_1 & = \sin  x_2 + (x_2 + 1) x_3\\
  \dot{x}_2 & = x^2_1 + x_3\\
  \dot{x}_3 & = x^2_2 + u\\
  y & = x_1
\end{align*}
对输出求一阶导数得到
\[ \dot{y} = \dot{x}_1 = \sin  x_2 + (x_2 + 1) x_3 \]
上式仍然不直接与$u$相关,尝试再求一阶导数得到
\begin{align*}
  \ddot{y} & = \dot{x}_2 \cos  x_2 + \dot{x}_2 x_3 + (x_2 + 1) \dot{x}_3\\
  & = (x^2_1 + x_3) \cos  x_2 + x_3 (x^2_1 + x_3) + (x_2 + 1) (x^2_2 + u)\\
  & = (x^2_1 + x_3) (\cos  x_2 + x_3) + (x_2 + 1) x^2_2 + u (x_2 + 1)\\
  & \triangleq  f(x) + g(x)u
\end{align*}
设计 \[u = \frac{1}{g(x)} (v - f (x))\]
我们就有 $\ddot{y} = v$。此时就可用设计线性系统的方法,设计 $v = - k_1 y - k_2 \dot{y},k_1,k_2>0$,即可使输出$y$稳定(提示:二阶微分方程$\ddot{y}+k_2 \dot{y}+k_1 y=0$的特征根具有负实部)。
\end{example}

\begin{remark}  
    这种方法的缺陷:
  \begin{enumerate}
    \item 需要得知所有的状态$(x_1, x_2, x_3)$才可计算控制$u$。
    
    \item $g(x)=0$,即$x_2 = - 1$时,$u$无法应用,即存在奇异点!
    
    \item $y$稳定了,但$x_2, x_3$ 未必稳定?
    \item 并不对所有非线性系统都适用。
  \end{enumerate}
\end{remark}

反馈线性化的基本思想,是将系统中的非线性项消去。然而,我们应该审慎考虑这一思想——{\bf 消去非线性项总是有利的吗}?

\begin{example}
考虑下述标量系统
    \[\dot{x} = a  x - x^3 + u\]
如果用反馈线性化的思想,那么应设计\[u=-ax+x^3+v\]
这样$\dot{x}=v$,利用$v=-kx,k>0$可镇定原系统。然而,如果一开始$x$离原点较远,$u$可能很大,即控制代价较大。

如果我们简单设计$u=-ax$,那么$\dot{x}=-x^3$,很容易看出系统已然渐近稳定。

因此,就像 \ref{1Aref} 节所说的那样,有些非线性对系统的运行是有害的,应设法克服它的有害影响;有些非线性是有益的,应在设计时予以考虑。后续章节的一些方法就会充分考虑这些有利因素(如 \ref{5Dref} 节要介绍的反步法)。
\end{example}
