\chapter{自适应控制}\label{cp4}
\section{概述}\label{4Aref}

% \subsubsection*{鲁棒控制(略)}

% \subsubsection*{增益调整(Gain Scheduling)}
% \addcontentsline{toc}{subsubsection}{增益调整(Gain Scheduling)}

% \subsubsection*{直接与非直接自适应控制}

% \subsubsection*{模型参考自适应控制(Model Reference Adaptive Control,MRAC)}
\begin{definition}[自适应控制器]
    \textbf{自适应控制器}是\textbf{结构固定}但具有\textbf{可变参数}的控制器,且带有一套自动调整参数的规则。
  \end{definition}
  
  \begin{description}
    \item[Why adaptive control?]
    \begin{enumerate}
      \item Systems to be controlled have {\textsf{parameter uncertainty}}.
      
      \item Systems experience {\textsf{unpredictable parameter variations}}.
      
      \item Unknown {\textsf{disturbance characteristics}}.
    \end{enumerate}
  \end{description}
  
  \begin{description}
    \item[Adaptive Control]
    \begin{enumerate}
      \item Adaptive control is superior to robust control in dealing with
      uncertainties in constant or slow-varying parameters.
      
      \item An adaptive controller improves its performance as adaptation goes
      on.
      
      \item An adaptive controller requires litter or no priori information
      about the unknown parameters.
    \end{enumerate}
  \end{description}
  
  \begin{description}
    \item[Robust Control]
    \begin{enumerate}
      \item Robust control has advantages in dealing with disturbance, quickly
      varying parameters and unmodeled dynamics.
      
      \item A robust controller attempts to keep consistent performance.
      
      \item A robust controller requires reasonable a prior estimates of the
      parameter bounds.
    \end{enumerate}
  \end{description}

\begin{enumerate}
    \item Characterize the desired behavior of the closed-loop systems.
    
    \item Determine a suitable control law containing adjustable parameters.
    
    \item Find a mechanism (an adaptation law) for adjusting those parameters.
    
    \item Analyze the convergence properties and implement the control law.
  \end{enumerate}
  \begin{description}
    \item[Gain Scheduling]
    \begin{enumerate}
      \item Controller parameters change in a predetermined fashion with the
      operating conditions.
      
      \item AE: the parameters can be changed quickly in response to changes in
      the dynamics.
      
      \item DE: it is an open-loop adaptation scheme, with no real ``learning''.
    \end{enumerate}
  \end{description}
  
  \begin{description}
    \item[Self-Tuning]
    \begin{enumerate}
      \item Controller parameters change in a predetermined fashion with the
      operating conditions.
      
      \item Performs simultaneous parameter identification and control.
      
      \item Uses Certainty Equivalence Principle
    \end{enumerate}
  \end{description}
  
  \begin{description}
    \item[Model Reference]
    \begin{enumerate}
      \item Plant: containing unknown parameters and having a known structure.
      
      \item RM: specifying the desired output of the control system.
      
      \item Feedback control law: containing adjustable parameters.
      
      \item Adaptation mechanism: updating the adjustable parameters.
    \end{enumerate}
  \end{description}
  
  \begin{description}
    \item[Types]
    \begin{enumerate}
      \item Indirected: estimate plant parameters $\Rightarrow$ design
      controller parameters. The process model and possibly the disturbance
      characteristics are first determined. The controller parameters are
      designed on the basis of this information.
      
      \item Directed: directly design controller parameter. The controller
      parameters are changed directly without the characteristics of the process
      and its disturbance first being determined.
    \end{enumerate}
  \end{description}