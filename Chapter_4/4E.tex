\section{鲁棒自适应控制(Robust adaptive control)}\label{4Eref}
\subsection{引入:参数漂移}
考虑下述标量系统
\begin{equation}
    \dot{x} = u + \varphi(x)\theta + d(t)\label{Sys:with_disturbance}
\end{equation}
其中$x$为状态,$u$为控制输入,$\varphi$是连续且已知的函数,$\theta$是未知常数,且$d(t)$是有界且时变的扰动,满足$|d(t)|\le d_{\max}$。

我们的控制目标是,设计$u$,使得闭环系统的所有信号都有界,且$x$趋近于原点或原点的一小邻域。

当$d(t)\equiv 0$时,应用 \ref{4Bref} 节的设计方法,采用下述自适应律
\[ u=-\varphi(x)\hat{\theta}(t)-kx\]
于是闭环系统动力学
\[\dot{x}=-kx-\varphi(x)\tilde{\theta}\]
其中$\tilde{\theta} \triangleq \hat{\theta}-\theta$。

考虑如下候选Lyapunov函数
\[V=\frac12 x^2+\frac{1}{2\gamma}\tilde{\theta}^2,\gamma>0\]
求其导数得
\begin{align*}
    \dot{V}&=x\cdot\dot{x}+\frac{1}{\gamma}\tilde{\theta}\dot{\hat{\theta}}
    =-kx^2+\frac{1}{\gamma}\tilde{\theta}(\dot{\hat{\theta}}-\gamma x\varphi(x))
\end{align*}
则设计
\begin{equation}
    \dot{\hat{\theta}}=\gamma x\varphi(x)\label{adaptive_nodisturbance}
\end{equation}
即可使得$\dot{V}=-kx^2$半负定,之后的分析类似 \ref{4Bref} 各小节。

然而,当$d(t)\ne 0$时,闭环系统动力学
\[\dot{x}=-kx-\varphi(x)\tilde{\theta}+d(t)\]
如果仍采用上述自适应律和候选Lyapunov函数,得到$V$的导数是
\[\dot{V}=-kx^2+x d(t)\le -kx^2+d_{\max}|x|=-k|x|(|x|-\frac{d_{\max}}{k})\]
则在下述集合
\[E = \left\{(x,\tilde{\theta}): \| x \| \leq \frac{d_{\max}}{k} = e_{0}\right\}\]
之外,才有$\dot{V}<0$,而在该集合之内,$\dot{V}$可能为正,因此,参数估计误差$|\tilde{\theta}(t)|$可能自由增长而最终变为无界。这种现象被称为{\bf 参数漂移(parameter drift)},是由干扰$d(t)$而产生的,这表明我们所提出的自适应律不具有抗干扰能力,即不是鲁棒的。

\begin{example}[扰动衰减但参数估计发散]
    假设\(\theta = k = \gamma = 1,\varphi(x) = x\)。则 \eqref{Sys:with_disturbance} 和 \eqref{adaptive_nodisturbance} 变为
\[\begin{cases}
\dot{x} = - x\hat{\theta} + d(t) \\
\dot{\hat{\theta}} = x^{2} \\
\end{cases}\]

若$x = (1 + t)^{- \frac{2}{5}} $,
此时 \eqref{adaptive_nodisturbance} 成为
$\dot{\hat{\theta}} = x^{2} = (1 + t)^{- \frac{4}{5}} $,
积分得$\hat{\theta}(t) = 5(1 + t)^{\frac{1}{5}},\hat{\theta}(0) = 5$。则\(x\hat{\theta}(t) = 5(1 + t)^{- \frac{1}{5}}\)。
对$x$求导得$\dot{x} = - \frac{2}{5}(1 + t)^{- \frac{7}{5}}$,因此
$d(t) = \dot{x} + x\hat{\theta} = 5(1 + t)^{- \frac{1}{5}} - \frac{2}{5}(1 + t)^{- \frac{7}{5}}$。

可见,随$t\to\infty$,上述\(d(t)\)收敛(意即扰动收敛),并且\(x\)收敛,然而,\(\hat{\theta}\)是发散的!
\end{example}

\subsection{死区修正(Dead-Zone Modification)}

为了提升鲁棒性,我们考虑下述死区修正自适应律

\[\dot{\hat{\theta}} = \left\{\begin{matrix}
\gamma x\varphi(x), & \text{若} \| x \| \geq e_{0} + \delta \\
0, & \text{若} \| x \| < e_{0} + \delta \\
\end{matrix}\right.\]

回顾\(V = \frac{1}{2}x^{2} + \frac{1}{2\gamma}{\tilde{\theta}}^{2}\)

定义\(\Omega_{1} = \{ t: \| x(t) \| < e_{0} + \delta\},\Omega_{2} = \{ t: \| x(t) \| \geq e_{0} + \delta\}\),当\(t \in \Omega_{2},\dot{V} \leq - k(e_{0} + \delta)\delta = - c < 0\)

(图示)

若\(t_{0} \in \Omega_{1}\),并且轨迹在\(t_{1}\)时刻到达边界\(e_{0} + \delta\),且在\((t_{1},t_{2}) \in \Omega_{2}\)处于\(e_{0} + \delta\)之右侧(越过边界)

\[V(t_{1}) = \frac{1}{2}(e_{0} + \delta)^{2} + \frac{1}{2\gamma}{\tilde{\theta}}^{2}(t_{0})\]

\[V(t_{2}) = \frac{1}{2}(e_{0} + \delta)^{2} + \frac{1}{2\gamma}{\tilde{\theta}}^{2}(t_{2}) < V(t_{1})\]

我们可推断出,\(\| \tilde{\theta}(t) \|\)在每个到达边界的时刻\(t_{1},t_{2},\ldots\)总会单调递减,因此\(\| \tilde{\theta}(t) \|\)是有界的,并且随时间增长一定有\(\| x \| \leq e_{0} + \delta\)。

从上面的内容中,不难看出,死区修正简单且有效,保证了参数的收敛。但是,死区修正还存在两个问题。一是,需要假设扰动的上界是先验已知的。二是,在无扰动的情形下并不能得到跟踪误差的渐近稳定性,换言之,鲁棒性的实现是以破坏自适应律的一些理想特性为代价的。关于后者,具体来说就是,我们只能确保跟踪误差进入死区($\| x \| \leq e_0 + \delta$),即使在无扰动的情形下也是如此,而不能像在无扰动且无死区(采用 \eqref{adaptive_nodisturbance})的情形下那样得到渐近稳定性。

\subsection{\texorpdfstring{$\sigma$-修正($\sigma$-Modification)}{σ-修正(σ-Modification)}}

\begin{equation} \label{sigma_modification}
    \dot{\hat{\theta}} = \gamma(x\varphi(x) - \sigma\hat{\theta})
\end{equation}

本质上,这种修正是向理想的自适应律上增加了一个阻尼项。动机就是阻碍参数漂移带来的影响。

在这种修正下,\(V(t)\)随时间,沿系统的导数是

\[\begin{aligned}
\dot{V} & = - kx^{2} - x\varphi(x)\tilde{\theta} + xd(t) + \frac{1}{\gamma}\tilde{\theta}\dot{\hat{\theta}} \\
 & = - kx^{2} - x\varphi(x)\tilde{\theta} + xd(t) + \tilde{\theta}x\varphi(x) - \sigma\tilde{\theta}\hat{\theta} \\
 & = - kx^{2} + xd(t) - \sigma\tilde{\theta}\hat{\theta} 
\end{aligned}\]

注意到\(xd(t) \leq d_{\max} \| x \| \leq \frac{k}{2}x^{2} + \frac{1}{2k}d_{\max}^{2}\)(利用不等式\(ab \leq \frac{ka^{2}}{2} + \frac{b^{2}}{2k}\))

目标形式是\(\dot{V} = - \beta V + C\)

并且\(- \sigma\tilde{\theta}\hat{\theta} = - \sigma\tilde{\theta}(\tilde{\theta} + \theta) \leq - \sigma\tilde{\theta}{\tilde{\theta}}^{2} + \sigma(\frac{{\tilde{\theta}}^{2}}{2} + \frac{\theta^{2}}{2}) = - \sigma\frac{{\tilde{\theta}}^{2}}{2} + \sigma\frac{\theta^{2}}{2}\)

将上面两个式子代入得到

\begin{align*}
\dot{V} & \leq - kx^{2} + \frac{k}{2}x^{2} + \frac{d_{\max}^{2}}{2k} - \sigma\frac{{\tilde{\theta}}^{2}}{2} + \sigma\frac{\theta^{2}}{2} \\
 & = - \frac{k}{2}x^{2} + \frac{d_{\max}^{2}}{2k} - \sigma\frac{{\tilde{\theta}}^{2}}{2} + \sigma\frac{\theta^{2}}{2} \\
 & = - \frac{k}{2}x^{2} - \sigma\gamma\frac{{\tilde{\theta}}^{2}}{2\gamma} + \frac{d_{\max}^{2}}{2k} + \sigma\frac{\theta^{2}}{2} \\
 & \leq - \min\{ k,\sigma\gamma\} V + \frac{d_{\max}^{2}}{2k} + \sigma\frac{\theta^{2}}{2}
\end{align*}

\[\dot{V} + \beta V \leq c\]

欲使\(\beta\)较大,那么可取\(k,\sigma\gamma\)都较大;欲使\(c\)较小,使得\(k\)较大且\(\sigma\)较小即可。

综合考虑,应该使\(k\)较大,\(\sigma\)较小,\(\gamma\)大得多。


\subsection{\texorpdfstring{$e$-修正($e$-Modification)}{e-修正(e-Modification)}}

应用 $\sigma$-修正会带来性能上的一些缺陷。当跟踪误差变得比较小时,自适应律的动力学 \eqref{sigma_modification} 近似为 $\dot{\hat{\theta}} = -\gamma \sigma\hat{\theta})$。因此,对于较小的跟踪误差,自适应参数 $\hat{\theta}$ 有返回到原点的趋势,也就是说,参数会将原来用以减小跟踪误差的增益值“清除”掉。

为了克服这些不符合期望的影响,Narendra 和 Annaswamy 提出了 $e$-修正。其主要思路是,将 \eqref{sigma_modification} 中恒定的阻尼增益 $\sigma$ 修正为与跟踪误差成比例。具体来说,对于系统 \eqref{Sys:with_disturbance},可将 \eqref{sigma_modification} 改为
\[
    \dot{\hat{\theta}} = \gamma \left( x\varphi(x) - \sigma \| x \| \hat{\theta} \right) \text{。}
\]

讲义上无具体推导,待补充。

\subsection{\texorpdfstring{自适应$\sigma$-修正(Adaptive $\sigma$-Modification)}{自适应σ-修正(Adaptive σ-Modification)}}

正如在上一小节所提到的,$\sigma$-修正可能会使自适应参数“有返回到原点的趋势”。而我们实际上希望自适应参数趋于真实参数 $\theta$。如果自适应参数的动力学为
\[
    \dot{\hat{\theta}} = \gamma \left( x\varphi(x) - \sigma \left(\hat{\theta} - \theta \right) \right)
\]
就好了。但是,参数的真值是未知的(如果已知的话,也不必如此折腾了)。于是,不妨沿用自适应控制的思路,对真值进行估计,得到如下自适应律
\[
\begin{cases}
    \dot{\hat{\theta}} = \gamma \left( x\varphi(x) - \sigma \left(\hat{\theta} - \hat{\theta_1} \right) \right) \\
    \dot{\hat{\theta_1}} = \delta \left( \hat{\theta} - \hat{\theta_1} \right), \quad \delta > 0 \text{。}
\end{cases}
\]

讲义上无具体推导,待补充。

\subsection{推广至多输入多输出(MIMO)系统与时变参数情形}

关于各种鲁棒自适应控制方法,我们前面只讨论了标量系统(以系统 \eqref{Sys:with_disturbance} 为例)。实际上,上述内容可以轻松推广至 MIMO 系统。只是在相应的 Lyapunov 分析中,需要将原来的标量相乘视情况改写为二次型或者改写为矩阵乘积的迹。

具体而言,我们可以考虑下述系统
\begin{equation}
    \dot{x} = A x + B \Lambda \left( u + \Theta^\mathrm{T} \Phi(x) \right) + d(t) \text{,}
\end{equation}
% 讲义上是 \Phi(t)
其中 $x \in \mathbf{R}^n$,控制输入 $u \in \mathbf{R}^m$,$A \in \mathbf{R}^{n \times n}$ 和 $B \in \mathbf{R}^{n \times m}$ 已知,未知的常值矩阵 $\Lambda \in \mathbf{R}^{m \times m}$ 是对角正定阵,$\Phi(x) \in \mathbf{R}^n$ 是连续且已知的函数,未知常值参数 $\Theta \in \mathbf{R}^{n \times m}$,$d(t) \in \mathbf{R}^n$ 是有界且时变的扰动($\| d(t) \| \leq d_{\mathrm{max}}$)。

下表来自前言中参考文献5的表 11.1(略有修改)。

\begin{table}[htbp]
  \centering
  \setcellgapes{4pt}
  \makegapedcells
  \caption{MIMO 系统的鲁棒 MRAC 设计}
  \begin{tabular}{p{4.0cm}p{10.0cm}}
  \hline
   开环被控对象 & $\dot{x} =  A x + B \Lambda \left( u + \Theta^\mathrm{T} \Phi(x) \right) + d(t)$\\
    参考模型 & $\dot{x}_{\ensuremath{\operatorname{ref}}} = 
    A_{\ensuremath{\operatorname{ref}}} x +
    B_{\ensuremath{\operatorname{ref}}} u_c$,其中$A_{\ensuremath{\operatorname{ref}}}$ 是 Hurwitz 的。\\
    跟踪误差 & $e = x - x_{\ensuremath{\operatorname{ref}}}$\\
    Lyapunov 公式 & $A_{\ensuremath{\operatorname{ref}}} + A_{\ensuremath{\operatorname{ref}}}^{\mathrm{T}} P = -Q < 0$\\
    控制输入 & $u = -{\hat{\Theta}}^{\mathrm{T}} \Phi(x)$\\
    死区修正 & $\dot{\hat{\Theta}} = \Gamma_\Theta \Phi(x) \mu\left( \| e \| \right) e^{\mathrm{T}} P B$\\
    $\sigma$-修正 & $\dot{\hat{\Theta}} = \Gamma_\Theta \Phi(x) \left( e^{\mathrm{T}} P B -\sigma \hat{\Theta} \right)$\\
    $e$-修正 & $\dot{\hat{\Theta}} = \Gamma_\Theta \Phi(x) \left( e^{\mathrm{T}} P B -\sigma \| e^{\mathrm{T}} P B \| \hat{\Theta} \right)$\\
    \hline
  \end{tabular}
\end{table}

具体推导见附录 \ref{MIMO_robust}。

至此,我们只讨论了常值参数自适应控制。下面我们简单讨论一下时变参数自适应控制。为简化讨论,我们仍以标量系统为例。考虑如下系统:
\begin{equation}
    \dot{x} = u + \varphi(x) \theta(t) \text{,}
\end{equation}
其中 $x$ 为状态,$u$ 为控制输入,$\phi$ 是已知的连续函数,$\theta(t)$ 为时变未知参数。我们采用控制输入
\[
    u = -k x - \varphi(x) \hat{\theta} \text{,}
\]
其中 $\hat{\theta}$ 的动力学方程随后给出。于是闭环系统动力学方程为
\[
    \dot{x} = -k x - \varphi(x) \tilde{\theta} \text{,}
\]
其中 $\tilde{\theta} \triangleq \hat{\theta} - \theta(t)$(由此, $\dot{\tilde{\theta}} = \dot{\hat{\theta}} - \dot{\theta}$)。

考虑如下候选 Lyapunov 函数
\[
    V = \dfrac{1}{2} x^2 + \dfrac{1}{2 \gamma} {\left( \hat{\theta} - \theta(t) \right)}^2, \quad \gamma > 0 \text{,}
\]
其沿系统轨线的函数为
\begin{align*}
    \dot{V} &= x \cdot \dot{x} + \dfrac{1}{\gamma} \tilde{\theta} \left( \dot{\hat{\theta}} - \dot{\theta} \right) \\
    &= -k x^2 - \varphi(x) x \tilde{\theta} + \dfrac{1}{\gamma} \tilde{\theta} \dot{\hat{\theta}} - \dfrac{1}{\gamma} \tilde{\theta} \dot{\theta} \text{。}
\end{align*}
$\hat{\theta}$ 的动力学方程采用 $\sigma$-修正,即式 \eqref{sigma_modification}。于是,
\[
    \dot{V} = -k x^2 - \sigma \tilde{\theta} \hat{\theta} - \dfrac{1}{\gamma} \tilde{\theta} \dot{\theta} \text{,}
\]
其中右侧的
\begin{gather*}
    -\sigma \tilde{\theta} \hat{\theta} \leq -\dfrac{\sigma}{2} {\tilde{\theta}}^2 + \dfrac{\sigma}{2} {\theta}^2, \\
    -\dfrac{1}{\gamma} \tilde{\theta} \dot{\theta} \leq \dfrac{m}{\gamma} \tilde{\theta} \leq \dfrac{\sigma}{4} \tilde{\theta}^2 + \dfrac{1}{\sigma} \dfrac{m^2}{\gamma^2} \text{,}
\end{gather*}
这里我们假设了 $\| \dot{\theta} \| \leq m$(即参数变化不会太快)。
从而,有
\[
    \dot{V} \leq -k x^2 - \dfrac{\sigma}{4} {\tilde{\theta}}^2 + \dfrac{\sigma}{2} \theta^2 + \dfrac{m^2}{\sigma \gamma^2} \text{。}
\]
类似地,我们可以得到形如 $\dot{V} + \beta{V} \leq c$(其中 $\beta$ 和 $c$ 为正的常值)的结果。

上面的推导中,我们假设了“参数变化不会太快”。而在参数变化较快的情况下,如果我们对于参数的范围能够有较好的估计,同样可以进行自适应控制的设计。假设参数 $\theta(t)$ 满足下式
\[
    \| \theta_\delta - \theta(t) \| \leq \delta \text{,}
\]
其中 $\theta_\delta$ 是常值,是对 $\theta$ 的估计。可采用如下自适应控制律
\begin{align*}
    u &= -k x - \varphi(x) \hat{\theta}, \\
    \dot{\hat{\theta}} &= \gamma x \varphi(x), \quad \gamma > 0 \text{。}
\end{align*}
则系统闭环动力学方程为
\[
    \dot{x} = -k x - \varphi(x) \left( \hat{\theta} - \theta(t) \right) \text{。}
\]
考虑如下候选 Lyapunov 函数
\[
    V = \dfrac{1}{2} x^2 + \frac{1}{2 \gamma} {\left( \hat{\theta} - \theta_\delta \right)}^2 \text{,}
\]
可得其沿系统轨线的导数
\begin{align*}
    \dot{V} &= -k x^2 - x \varphi(x) \left( \hat{\theta} - \theta(t) \right) + \dfrac{1}{\gamma} \left( \hat{\theta} - \theta_\delta \right) \cdot \dot{\hat{\theta}} \\
    &= -k x^2 - x \varphi(x) \left( \hat{\theta} - \theta(t) \right) +  \left( \hat{\theta} - \theta_\delta \right) \cdot x \varphi(x) \\
    &= -k x^2 + x \varphi(x) \left( \theta(t) - \theta_\delta \right) \\
    &\leq -k x^2 + \delta x \varphi(x) \text{。}
\end{align*}
% 讲义上到此为止,后续或有补充。
