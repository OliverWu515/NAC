\section{鲁棒自适应控制(Robust adaptive control)}\label{4Eref}
\subsection{参数漂移}
考虑下述标量系统
\begin{equation}
    \dot{x} = u + \varphi(x)\theta + d(t)\label{Sys:with_disturbance}
\end{equation}
其中$x$为状态,$u$为控制输入,$\varphi$是连续且已知的函数,$\theta$是未知常数,且$d(t)$是有界且时变的扰动,满足$|d(t)|\le d_{\max}$。

我们的控制目标是,设计$u$,使得闭环系统的所有信号都有界,且$x$趋近于原点或原点的一小邻域。

当$d(t)\equiv 0$时,应用 \ref{4Bref} 节的设计方法,采用下述自适应律
\[ u=-\varphi(x)\hat{\theta}(t)-kx\]
于是闭环系统动力学
\[\dot{x}=-kx-\varphi(x)\tilde{\theta}\]
其中$\tilde{\theta} \triangleq \hat{\theta}-\theta$。

考虑如下候选Lyapunov函数
\[V=\frac12 x^2+\frac{1}{2\gamma}\tilde{\theta}^2,\gamma>0\]
求其导数得
\begin{align*}
    \dot{V}&=x\cdot\dot{x}+\frac{1}{\gamma}\tilde{\theta}\dot{\hat{\theta}}\\
    &=-kx^2+\frac{1}{\gamma}\tilde{\theta}(\dot{\hat{\theta}}-\gamma x\varphi(x))
\end{align*}
则设计
\begin{equation}
    \dot{\hat{\theta}}=\gamma x\varphi(x)\label{adaptive_nodisturbance}
\end{equation}
即可使得$\dot{V}=-kx^2$半负定,之后的分析类似 \ref{4Bref} 各小节。

然而,当$d(t)\ne 0$时,闭环系统动力学
\[\dot{x}=-kx-\varphi(x)\tilde{\theta}+d(t)\]
如果仍采用上述自适应律和候选Lyapunov函数,得到$V$的导数是
\[\dot{V}=-kx^2+x d(t)\le -kx^2+d_{\max}|x|=-k|x|(|x|-\frac{d_{\max}}{k})\]
则在下述集合
\[E = \{(x,\tilde{\theta}): \| x \| \leq \frac{d_{\max}}{k} = e_{0}\}\]
之外,才有$\dot{V}<0$,而在该集合之内,$\dot{V}$可能为正,因此,参数估计误差$|\tilde{\theta}(t)|$可能自由增长而最终变为无界。这种现象被称为{\bf 参数漂移(parameter drift)},是由干扰$d(t)$而产生的,这表明我们所提出的自适应律不具有抗干扰能力,即不是鲁棒的。

\begin{example}[扰动衰减但参数估计发散]
    假设\(\theta = k = \gamma = 1,\varphi(x) = x\)。则 \eqref{Sys:with_disturbance} 变为
\[\begin{cases}
\dot{x} = - x\hat{\theta} + d(t) \\
\dot{\hat{\theta}} = x^{2} \\
\end{cases}\]

考虑$x = (1 + t)^{- \frac{2}{5}} $的情况。
此时 \eqref{adaptive_nodisturbance} 成为
\[\dot{\hat{\theta}} = x^{2} = (1 + t)^{- \frac{4}{5}} \]
积分得$\hat{\theta}(t) = 5(1 + t)^{\frac{1}{5}},\hat{\theta}(0) = 5$。则\(x\hat{\theta}(t) = 5(1 + t)^{- \frac{1}{5}}\)。

对$x$求导得$\dot{x} = - \frac{2}{5}(1 + t)^{- \frac{7}{5}}$,因此
\[d(t) = \dot{x} + x\hat{\theta} = 5(1 + t)^{- \frac{1}{5}} - \frac{2}{5}(1 + t)^{- \frac{7}{5}}\]

可见,上述\(d(t)\)收敛(意即扰动收敛),并且\(x\)收敛,然而,\(\hat{\theta}\)是发散的!
\end{example}

\subsection{死区修正(Dead-Zone Modification)}

为了提升鲁棒性,我们考虑下述死区修正自适应律

\[\dot{\hat{\theta}} = \{\begin{matrix}
\gamma x\varphi(x), & \text{若} \parallel x \parallel \geq e_{0} + \delta \\
0, & \text{若} \parallel x \parallel < e_{0} + \delta \\
\end{matrix}\]

回顾\(V = \frac{1}{2}x^{2} + \frac{1}{2\gamma}{\tilde{\theta}}^{2}\)

定义\(\Omega_{1} = \{ t: \parallel x(t) \parallel < e_{0} + \delta\},\Omega_{2} = \{ t: \parallel x(t) \parallel \geq e_{0} + \delta\}\),当\(t \in \Omega_{2},\dot{V} \leq - k(e_{0} + \delta)\delta = - c < 0\)

(图示)

若\(t_{0} \in \Omega_{1}\),并且轨迹在\(t_{1}\)时刻到达边界\(e_{0} + \delta\),且在\((t_{1},t_{2}) \in \Omega_{2}\)处于\(e_{0} + \delta\)之右侧(越过边界)

\[V(t_{1}) = \frac{1}{2}(e_{0} + \delta)^{2} + \frac{1}{2\gamma}{\tilde{\theta}}^{2}(t_{0})\]

\[V(t_{2}) = \frac{1}{2}(e_{0} + \delta)^{2} + \frac{1}{2\gamma}{\tilde{\theta}}^{2}(t_{2}) < V(t_{1})\]

我们可推断出,\(\parallel \tilde{\theta}(t) \parallel\)在每个到达边界的时刻\(t_{1},t_{2},\ldots\)总会单调递减,因此\(\parallel \tilde{\theta}(t) \parallel\)是有界的,并且随时间增长一定有\(\parallel x \parallel \leq e_{0} + \delta\)。

\subsection{\texorpdfstring{$\sigma$-修正($\sigma$-Modification)}{σ-修正(σ-Modification)}}

\[\dot{\hat{\theta}} = \gamma(x\varphi(x) - \sigma\hat{\theta})\]

本质上,这种修正是向理想的自适应律上增加了一个阻尼项。动机就是阻碍参数漂移带来的影响。

在这种修正下,\(V(t)\)随时间,沿系统的导数是

\[\begin{matrix}
\dot{V} & = - kx^{2} - x\varphi(x)\tilde{\theta} + xd(t) + \frac{1}{\gamma}\tilde{\theta}\dot{\hat{\theta}} \\
 & = - kx^{2} - x\varphi(x)\tilde{\theta} + xd(t) + \tilde{\theta}x\varphi(x) - \sigma\tilde{\theta}\hat{\theta} \\
 & = - kx^{2} + xd(t) - \sigma\tilde{\theta}\hat{\theta} \\
\end{matrix}\]

注意到\(xd(t) \leq d_{\max} \parallel x \parallel \leq \frac{k}{2}x^{2} + \frac{1}{2k}d_{\max}^{2}\)(利用不等式\(ab \leq \frac{ka^{2}}{2} + \frac{b^{2}}{2k}\))

目标形式是\(\dot{V} = - \beta V + C\)

并且\(- \sigma\tilde{\theta}\hat{\theta} = - \sigma\tilde{\theta}(\tilde{\theta} + \theta) \leq - \sigma\tilde{\theta}{\tilde{\theta}}^{2} + \sigma(\frac{{\tilde{\theta}}^{2}}{2} + \frac{\theta^{2}}{2}) = - \sigma\frac{{\tilde{\theta}}^{2}}{2} + \sigma\frac{\theta^{2}}{2}\)

将上面两个式子代入得到

\begin{align*}
\dot{V} & \leq - kx^{2} + \frac{k}{2}x^{2} + \frac{d_{\max}^{2}}{2k} - \sigma\frac{{\tilde{\theta}}^{2}}{2} + \sigma\frac{\theta^{2}}{2} \\
 & = - \frac{k}{2}x^{2} + \frac{d_{\max}^{2}}{2k} - \sigma\frac{{\tilde{\theta}}^{2}}{2} + \sigma\frac{\theta^{2}}{2} \\
 & = - \frac{k}{2}x^{2} - \sigma\gamma\frac{{\tilde{\theta}}^{2}}{2\gamma} + \frac{d_{\max}^{2}}{2k} + \sigma\frac{\theta^{2}}{2} \\
 & \leq - \min\{ k,\sigma\gamma\} V + \frac{d_{\max}^{2}}{2k} + \sigma\frac{\theta^{2}}{2}
\end{align*}

\[\dot{V} + \beta V \leq c\]

欲使\(\beta\)较大,那么可取\(k,\sigma\gamma\)都较大;欲使\(c\)较小,使得\(k\)较大且\(\sigma\)较小即可。

综合考虑,应该使\(k\)较大,\(\sigma\)较小,\(\gamma\)大得多。


\subsection{\texorpdfstring{$e$-修正($e$-Modification)}{e-修正(e-Modification)}}

\[\dot{\hat{\theta}} = \gamma(x\varphi(x) - \sigma \parallel x \parallel \hat{\theta})\]

\subsection{\texorpdfstring{自适应$\sigma$-修正(Adaptive $\sigma$-Modification)}{自适应σ-修正(Adaptive σ-Modification)}}