\section{多输入多输出(MIMO)系统的模型参考自适应控制}

\begin{equation}\label{MIMO}
  \dot{x}=Ax+B\Lambda u
\end{equation}

其中$x\in\R{}^n$是系统的状态,

\begin{align}
  \dot{e}&=\dot{x}-\dot{x}_{ref}\nonumber\\
  &=A_{ref}e+B\Lambda\tilde{K}_1x+B\Lambda\tilde{K}_2u_c
\end{align}

类比之前的构造,考虑如下“形式上的”Lyapunov函数
\begin{align*}
  V&=\frac12e^2+\frac{1}{2}\tilde{K}_1^2+\frac{1}{2}\tilde{K}_2^2
\end{align*}
由于其中的$e,\tilde{K}_1,\tilde{K}_2$都是向量或矩阵,而Lyapunov函数输出是标量,上述式子要做相应的修改。
首先,将$e^2$换成$e^\mathrm{T}e$;然后,将$\tilde{K}_1^2$换为\[\tr(\tilde{K}_1^\mathrm{T}\tilde{K}_1)\]
由于$\tilde{K}_1^\mathrm{T}\tilde{K}_1$是实对称矩阵,故其一定可做相似对角化,对角化之后的,而迹是相似不变量,

半正定的,;
而迹等于矩阵所有特征值之和,故$\tilde{K}_1\ne0$时,上式一定为正;

\begin{equation}\label{MIMOLYA}
  V=e^\mathrm{T}Pe+\tr(\tilde{K}_1^\mathrm{T}\tilde{K}_1)+\tr(\tilde{K}_2^\mathrm{T}\tilde{K}_2)
\end{equation}

其中$P$是满足如下Lyapunov方程的正定矩阵。求其对$t$的导数可得
\begin{align}
  \dot{V}&=\dot{e}^\mathrm{T}Pe+e^\mathrm{T}P\dot{e}+2\tr(\tilde{K}_1^\mathrm{T}\dot{\tilde{K}}_1)+
  2\tr(\tilde{K}_2^\mathrm{T}\dot{\tilde{K}}_2)\\
  &=\dot{e}^\mathrm{T}(A_{ref}^\mathrm{T}P+PA_{ref})e+2{e}^\mathrm{T}PB\Lambda\tilde{K}_1x+2{e}^\mathrm{T}PB\Lambda
  \tilde{K}_2u_c+2\tr(\tilde{K}_2^\mathrm{T}\dot{\tilde{K}}_2)\\
  &=-e^\mathrm{T}Qe+2\tr(\tilde{K}_1^\mathrm{T}\Lambda B^\mathrm{T}Pex^\mathrm{T})+2\tr(\tilde{K}_2^\mathrm{T}\Lambda B^\mathrm{T}Peu_c^\mathrm{T})+2{e}^\mathrm{T}PB\Lambda\tilde{K}_2u_c+2\tr(\tilde{K}_2^\mathrm{T}\dot{\hat{K}}_2)
\end{align}
其中第三行我们用到了
\begin{align*}
  {e}^\mathrm{T}PB\Lambda\tilde{K}_1x&=\tr({e}^\mathrm{T}PB\Lambda\tilde{K}_1x)&\text{(标量)}\\
  &=\tr(x{e}^\mathrm{T}PB\Lambda\tilde{K}_1)&\text{(}\tr(AB)=\tr(BA)\text{)}\\
  &=\tr(\tilde{K}_1^\mathrm{T}\Lambda B^\mathrm{T}Pex^\mathrm{T})&\text{(}\tr(A^\mathrm{T})=\tr(A)\text{)}
\end{align*}
于是我们可设计

但是该设计是不可行的,

\begin{equation}\label{MODMIMOLYA}
  V=e^\mathrm{T}Pe+\tr(\tilde{K}_1^\mathrm{T}\textcolor{red}{\Lambda}\tilde{K}_1)+\tr(\tilde{K}_2^\mathrm{T}\textcolor{red}{\Lambda}\tilde{K}_2)
\end{equation}
