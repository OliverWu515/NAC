\section{多输入多输出(MIMO)系统的模型参考自适应控制}\label{4Cref}
考虑下述线性系统
\begin{equation}
  \dot{x} = A  x + B \Lambda u\label{Sys:MRAC:MIMO}
\end{equation}
其中 $x \in \mathbf{R}^n$是系统的状态, $u \in \mathbf{R}^m$ 是控制输入,$B \in \mathbf{R}^{n \times m}$ 是已知的控制矩阵,而 $A \in \mathbf{R}^{n \times n}, \Lambda \in \mathbf{R}^{m \times m}$ 是未知的常值矩阵。又假设  $\Lambda$ 是对角阵,且对角线上均为正数,且$(A, B \Lambda)$ 可控。$\Lambda$是不确定的,这是用于刻画控制失效或建模不准确,意即控制增益可能不确定,或是设计者估计的系统控制效果可能是不准确的。

{\bf 控制目标是:}设计$u$,使得闭环系统的所有信号都有界,且$x$能够跟踪下述参考模型的参考状态 $x_{\ensuremath{\operatorname{ref}}}$:
 \begin{equation}
    \dot{x}_{\ensuremath{\operatorname{ref}}} =
    A_{\ensuremath{\operatorname{ref}}} x_{\ensuremath{\operatorname{ref}}} +
    B_{\ensuremath{\operatorname{ref}}} u_c\label{ideal_mimo}
  \end{equation}
其中$A_{\ensuremath{\operatorname{ref}}} \in \mathbf{R}^{n \times n}$ 是Hurwitz的,$B_{\ensuremath{\operatorname{ref}}} \in \mathbf{R}^{n \times m}$, $u_c \in \mathbf{R}^m$ 是有界的(虚拟)输入指令。

首先推导理想情形。若$A \in \mathbf{R}^{n \times n}$ 与 $\Lambda \in \mathbf{R}^{m\times m}$ 已知,我们可选用如下控制律
\[ u = K^{\ast}_1 x + K^{\ast}_2 u_c \]
其中 $K^{\ast}_1 \in \mathbf{R}^{m \times n}, K^{\ast}_2 \in \mathbf{R}^{m
\times m}$,将上式代入 \eqref{Sys:MRAC:MIMO} 可得
\[ \dot{x} = (A + B \Lambda K^{\ast}_1) x + B \Lambda K^{\ast}_2 u_c \]
与理想情形对比可知
\begin{equation}
  \left\{\begin{array}{l}
    A + B \Lambda K^{\ast}_1 = A_{\ensuremath{\operatorname{ref}}}\\
    B \Lambda K^{\ast}_2 = B_{\ensuremath{\operatorname{ref}}}
  \end{array}\right. \label{matching:condition:GLS}
\end{equation}
我们应注意,一般地,无法保证满足 \eqref{matching:condition:GLS} 的理想增益阵$K^{\ast}_1, K^{\ast}_2$存在。不过,在实际应用中,如果$A, B$的结构已知,可以设计 $A_{\ensuremath{\operatorname{ref}}}, B_{\ensuremath{\operatorname{ref}}}$ 以保证 \eqref{matching:condition:GLS} 有解。

下面假设满足 \eqref{matching:condition:GLS} 的$K^{\ast}_1, K^{\ast}_2$存在(there is sufficient structure flexibility to meet the control
objective)。由于$\Lambda$未知,我们提出以下含可变矩阵参数的控制律
\[ u = \hat{K}_1 (t) x + \hat{K}_2 (t) u_c \]
代入 \eqref{Sys:MRAC:MIMO} 可得
\begin{align}
    \dot{x}&=A  x + B \Lambda (\hat{K}_1 (t) x + \hat{K}_2 (t) u_c)+
    \underbrace{A_{\ensuremath{\operatorname{ref}}} x +
    B_{\ensuremath{\operatorname{ref}}} u_c-A_{\ensuremath{\operatorname{ref}}} x -
    B_{\ensuremath{\operatorname{ref}}} u_c}_{\text{增减期望形式}}\nonumber\\
    &=A_{\ensuremath{\operatorname{ref}}} x +
    B_{\ensuremath{\operatorname{ref}}} u_c+(A + B \Lambda \hat{K}_1 (t)-A_{\ensuremath{\operatorname{ref}}}) x+(B\Lambda\hat{K}_2 (t)-B_{\ensuremath{\operatorname{ref}}}) u_c\nonumber\\
    &=A_{\ensuremath{\operatorname{ref}}} x +
    B_{\ensuremath{\operatorname{ref}}} u_c+(B \Lambda \hat{K}_1 (t)-B \Lambda K^{\ast}_1) x+(B\Lambda\hat{K}_2 (t)-B \Lambda K^{\ast}_2) u_c\label{mimo_dotx}
\end{align}
第三行是将 \eqref{matching:condition:GLS} 代入。记参数估计误差 $\tilde{K}_1 \triangleq \hat{K}_1 {- K_1^{\ast}} , \tilde{K}_2
\triangleq \hat{K}_2 {- K_2^{\ast}} $,则
\[ \dot{x} = A_{\ensuremath{\operatorname{ref}}}  x +
   B_{\ensuremath{\operatorname{ref}}} u_c + B \Lambda  \tilde{K}_1 x + B
   \Lambda  \tilde{K}_2 u_c \]

定义跟踪误差 $e \triangleq x -
x_{\ensuremath{\operatorname{ref}}}$。根据 \eqref{ideal_mimo} 和 \eqref{mimo_dotx},写出误差动力学
\begin{equation}
  \dot{e} = A_{\ensuremath{\operatorname{ref}}} e + B \Lambda  \tilde{K}_1 x +
  B \Lambda  \tilde{K}_2 u_c \label{errsys:MRAC:GLS}
\end{equation}
类比之前的构造,考虑如下“形式上的”Lyapunov函数
\begin{align*}
  V&=\frac12e^2+\frac{1}{2}\tilde{K}_1^2+\frac{1}{2}\tilde{K}_2^2
\end{align*}
由于其中的$e,\tilde{K}_1,\tilde{K}_2$都是向量或矩阵,而Lyapunov函数输出是标量,上式要做相应的修改。

首先,将$\frac12e^2$换成$\frac12e^\mathrm{T}e$。求其导数得\[e^\mathrm{T}\dot{e}=e^\mathrm{T}A_{\ensuremath{\operatorname{ref}}} e+\dots\]
省略的项待与后面的项合并。仅考察第一项,我们发现由于$A_{\ensuremath{\operatorname{ref}}}$未必负定,则第一项并不保证为负定。后续若要类似之前思路保留该项则不可行。考虑重新设计此项为\[e^\mathrm{T}\textcolor{red}{P}e\]
其中$P$为正定阵。则此项导数是
\[e^\mathrm{T}P\dot{e}+\dot{e}^\mathrm{T}Pe=e^\mathrm{T}(A_{\ensuremath{\operatorname{ref}}}^\mathrm{T} P + P A_{\ensuremath{\operatorname{ref}}}) e+\dots\]
由于 $A_{\ensuremath{\operatorname{ref}}}$ 是 Hurwitz的,根据定理 \ref{linearlya},我们可知对任意正定矩阵 $Q \in \mathbf{R}^{n \times
n}$,都存在唯一正定阵 $P \in \mathbf{R}^{n \times n}$
使得
\[ A_{\ensuremath{\operatorname{ref}}}^\mathrm{T} P + P 
   A_{\ensuremath{\operatorname{ref}}} = - Q < 0 \]
则第一项必为负定。

然后,将$\tilde{K}_1^2$换为\[\tr(\tilde{K}_1^\mathrm{T}\tilde{K}_1)\]
由 \ref{pd_npd_div},$\tilde{K}_1^\mathrm{T}\tilde{K}_1$ 是半正定的,进而由 \ref{pd_nd_thm},其特征值全为非负,则其迹一定非负,则上式是半正定的。

下面说明$\tr(\tilde{K}_1^\mathrm{T}\tilde{K}_1)=0\iff \tilde{K}_1=0$。仅证“$\implies$”情形。由于$\tilde{K}_1^\mathrm{T}\tilde{K}_1$特征值非负,所以迹为$0$就表明所有特征值均为$0$。则由 \ref{ortho_sim_diag},存在正交矩阵$S$使得$S^{-1}\tilde{K}_1^\mathrm{T}\tilde{K}_1S=0$。则$\tilde{K}_1^\mathrm{T}\tilde{K}_1=0$。因此对于所有$x\in\mathbf{R}^{n}$均有$x^{\mathrm{T}}\tilde{K}_1^\mathrm{T}\tilde{K}_1x=0$,因此$\tilde{K}_1x=0$,则$\tilde{K}_1=0$。因此,$\tr(\tilde{K}_1^\mathrm{T}\tilde{K}_1)$正定。

类似将$\tilde{K}_2^2$换为$\tr(\tilde{K}_2^\mathrm{T}\tilde{K}_2)$。
于是考虑如下候选Lyapunov函数
\begin{equation}\label{MODMIMOLYA:modify1}
V=e^\mathrm{T}Pe+\tr(\tilde{K}_1^\mathrm{T}\tilde{K}_1)+\tr(\tilde{K}_2^\mathrm{T}\tilde{K}_2)
\end{equation}

求其导数得(注意$\dot{\tilde{K}}_1=\dot{\hat{K}}_1$,$\dot{\tilde{K}}_2=\dot{\hat{K}}_2$)
\begin{align*}
  \dot{V} & =e^\mathrm{T}P\dot{e}+\dot{e}^\mathrm{T}Pe + 2\ensuremath{\operatorname{tr}} \{
  \tilde{K}^\mathrm{T}_1 \dot{\hat{K}} _1 \} + 2\ensuremath{\operatorname{tr}}
  \{ \tilde{K}^\mathrm{T}_2 \dot{\hat{K}} _2 \}\\
  & = e^\mathrm{T}P( A_{\ensuremath{\operatorname{ref}}} e + B \Lambda  \tilde{K}_1 x +
  B \Lambda  \tilde{K}_2 u_c)+(A_{\ensuremath{\operatorname{ref}}} e + B \Lambda  \tilde{K}_1 x +
  B \Lambda  \tilde{K}_2 u_c)^\mathrm{T}Pe\\
  &\quad+
  2\ensuremath{\operatorname{tr}} \{ \tilde{K}^\mathrm{T}_1 \dot{\hat{K}} _1 \}
  + 2\ensuremath{\operatorname{tr}} \{ \tilde{K}^\mathrm{T}_2 \dot{\hat{K}} _2 \}\\
  & = e^\mathrm{T}(A_{\ensuremath{\operatorname{ref}}}^\mathrm{T} P + P A_{\ensuremath{\operatorname{ref}}}) e+{e}^\mathrm{T}PB\Lambda\tilde{K}_1x+(B\Lambda \tilde{K}_1x)^\mathrm{T}Pe+{e}^\mathrm{T}PB\Lambda
  \tilde{K}_2u_c\\
  &\quad +(B\Lambda \tilde{K}_2u_c)^\mathrm{T}Pe+
  2\ensuremath{\operatorname{tr}} \{ \tilde{K}^\mathrm{T}_1 \dot{\hat{K}} _1 \}
  + 2\ensuremath{\operatorname{tr}} \{ \tilde{K}^\mathrm{T}_2 \dot{\hat{K}} _2 \}
\end{align*}
利用${e}^\mathrm{T}PB\Lambda\tilde{K}_1x$为标量且$P$为对称,有
\[{e}^\mathrm{T}PB\Lambda\tilde{K}_1x=({e}^\mathrm{T}PB\Lambda\tilde{K}_1x)^\mathrm{T}=(B\Lambda\tilde{K}_1x)^\mathrm{T}P^\mathrm{T}(e^\mathrm{T})^\mathrm{T}=(B\Lambda\tilde{K}_1x)^\mathrm{T}Pe\]
类似处理${e}^\mathrm{T}PB\Lambda\tilde{K}_2u_c$。则
\begin{align}
 \dot{V} & = e^\mathrm{T}(A_{\ensuremath{\operatorname{ref}}}^\mathrm{T} P + P A_{\ensuremath{\operatorname{ref}}}) e+ 2 e^\mathrm{T}  P  B \Lambda 
  \tilde{K}_1 x + 2 e^\mathrm{T} P  B \Lambda  \tilde{K}_2 u_c +
  2\ensuremath{\operatorname{tr}} \{ \tilde{K}^\mathrm{T}_1  \dot{\hat{K}} _1 \}
  + 2\ensuremath{\operatorname{tr}} \{ \tilde{K}^\mathrm{T}_2  \dot{\hat{K}} _2
  \}\nonumber\\
  & = -e^\mathrm{T}Q e +
  2\ensuremath{\operatorname{tr}} \{ \tilde{K}^\mathrm{T}_1 \Lambda B^\mathrm{T}  P  e  x^\mathrm{T} \} +
  2\ensuremath{\operatorname{tr}} \{ \tilde{K}^\mathrm{T}_2 \Lambda B^\mathrm{T}  P  e  u_c^\mathrm{T} \} +
  2\ensuremath{\operatorname{tr}} \{ \tilde{K}^\mathrm{T}_1  \dot{\hat{K}} _1 \}
  + 2\ensuremath{\operatorname{tr}} \{ \tilde{K}^\mathrm{T}_2  \dot{\hat{K}} _2
  \}\label{dot_V_modify1}
\end{align}
其中第二行我们用到了
\begin{equation}
    \begin{aligned}
 {e}^\mathrm{T}PB\Lambda\tilde{K}_1x&=\tr({e}^\mathrm{T}PB\Lambda\tilde{K}_1x)&\text{(标量)}\\
  &=\tr(x{e}^\mathrm{T}PB\Lambda\tilde{K}_1)&\text{(}\tr(AB)=\tr(BA)\text{)}\\
  &=\tr(\tilde{K}_1^\mathrm{T}\Lambda B^\mathrm{T}Pex^\mathrm{T})&\text{(}\tr(A^\mathrm{T})=\tr(A),\Lambda,P\ \text{对称)}
    \end{aligned}\label{trace_transform}
\end{equation}
对含$x_c$项的处理类似。尝试选取
\[\dot{\hat{K}} _1 = - \Lambda B^\mathrm{T}  P  e  x^\mathrm{T}, \dot{\hat{K}} _2 = - \Lambda B^\mathrm{T}  P  e  u^\mathrm{T}_c\]
以消去 \eqref{dot_V_modify1} 最后四项。然而,$\Lambda$未知。因此,该设计是不可行的。

将 \eqref{MODMIMOLYA:modify1} 修改为
\begin{equation}\label{MODMIMOLYA:modify2}
  V=e^\mathrm{T}Pe+\tr(\tilde{K}_1^\mathrm{T}\textcolor{red}{\Lambda}\tilde{K}_1)+\tr(\tilde{K}_2^\mathrm{T}\textcolor{red}{\Lambda}\tilde{K}_2)
\end{equation}
由于$\Lambda$正定,所以上述第二、三项的正定性不变(提示:可拆分为$\tilde{K}_1^\mathrm{T}\Lambda^\frac{1}{2}\Lambda^\frac{1}{2}\tilde{K}_1$)。因此,该修改是可行的。

此时 \eqref{dot_V_modify1} 变为
\begin{align*}
  \dot{V} &= -e^\mathrm{T}Q e +
  2\ensuremath{\operatorname{tr}} \{ \tilde{K}^\mathrm{T}_1 \Lambda B^\mathrm{T}  P  e  x^\mathrm{T} \} +
  2\ensuremath{\operatorname{tr}} \{ \tilde{K}^\mathrm{T}_2 \Lambda B^\mathrm{T}  P  e  u_c^\mathrm{T} \} +
  2\ensuremath{\operatorname{tr}} \{ \tilde{K}^\mathrm{T}_1 \textcolor{red}{\Lambda} \dot{\hat{K}} _1 \}
  + 2\ensuremath{\operatorname{tr}} \{ \tilde{K}^\mathrm{T}_2 \textcolor{red}{\Lambda} \dot{\hat{K}} _2 \}
\end{align*}

于是我们可选取
\[\dot{\hat{K}} _1 = - B^\mathrm{T}  P  e  x^\mathrm{T}, \dot{\hat{K}} _2 = - B^\mathrm{T}  P  e  u^\mathrm{T}_c\]
这样就有
\[ \dot{V} = - e^\mathrm{T} Q  e \leq 0 \]
于是$V (t) \leq V (0)$,$V$有界,由 \eqref{MODMIMOLYA:modify2} 的形式可知 $e, \tilde{K}_1, \tilde{K}_2 \in
\mathbf{L}_{\infty}$。由于$u_c$ 有界且
$A_{\ensuremath{\operatorname{ref}}}$ 是 Hurwitz的,所以
$x_{\ensuremath{\operatorname{ref}}} \in \mathbf{L}_{\infty}$。又知 $x = e +x_{\ensuremath{\operatorname{ref}}}$,所以$x \in \mathbf{L}_{\infty}$。根据 \eqref{errsys:MRAC:GLS},$\dot{e} \in \mathbf{L}_{\infty}$。对$V$再求一阶导数,有
\[ \ddot{V} = - 2 e^\mathrm{T} Q  \dot{e} \in \mathbf{L}_{\infty} \]
用 Barbalat引理(\ref{barbalat}),有$\lim\limits_{t \rightarrow \infty} \dot{V} = 0$,也即$\lim\limits_{t \rightarrow \infty} e (t) = 0$。

\begin{problem}
    $\Lambda$ 对角线元素全为负?
\end{problem}
\begin{hint}
\eqref{MODMIMOLYA:modify2} 第二项改为 $\ensuremath{\operatorname{tr}} \{ \tilde{K}^\mathrm{T}_1 (- \Lambda) \tilde{K} _1 \}$,第三项类似。
\end{hint}
\begin{problem}
    $\Lambda$ 对角线上的元素有些为正、有些为负?
\end{problem}
\begin{hint}
    选取\[ V = e^\mathrm{T} P  e +\ensuremath{\operatorname{tr}} \{ \tilde{K}^\mathrm{T}_1 |
       \Lambda | \tilde{K} _1 \} +\ensuremath{\operatorname{tr}} \{
       \tilde{K}^\mathrm{T}_2 | \Lambda | \tilde{K} _2 \} \]
    则可得
    \[ \dot{\hat{K}} _1 = -\ensuremath{\operatorname{sgn}} (\Lambda) B^\mathrm{T}  P  e
       x^\mathrm{T}, \dot{\hat{K}} _2 = -\ensuremath{\operatorname{sgn}} (\Lambda) B^\mathrm{T}
       P  e  u^\mathrm{T}_c \]
    其中 $| \Lambda | \triangleq \Lambda \ensuremath{\operatorname{sgn}}
    (\Lambda)$, $\ensuremath{\operatorname{sgn}} (\Lambda) \triangleq
    \ensuremath{\operatorname{diag}} \{ \ensuremath{\operatorname{sgn}}
    (\lambda_i) \}$。
\end{hint}

\begin{problem}\label{Pro:uncertainty}
系统方程变为
  \begin{equation}
      \dot{x} = A  x + B \Lambda (u + \Theta\Phi (t))\label{Sys:MRAC:MIMO_with_dis}
  \end{equation}
  其中 $\Theta \in \mathbf{R}^{m \times n}, \Phi (t) \in \mathbf{R}^{n \times 1}$,$\Theta$ 未知,$\Phi$ 已知且有界。
\end{problem}
\begin{hint}
    设计控制律\[ u = \hat{K}_1 (t) x + \hat{K}_2 (t) u_c - \hat{\Theta}\Phi (t)  \]
\end{hint}