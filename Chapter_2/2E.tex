\section{线性系统与线性化}\label{2Eref}
线性系统中,对应于非线性自治系统的是线性定常(线性时不变,Linear Time-Invariant,简写作LTI)系统
\begin{equation}\label{LTI}
    \dot{x}=Ax
\end{equation}

  Lyapunov分析法使我们得以在同一框架下研究线性和非线性系统(将LTI视为非线性系统的一个特例)。
 接下来,我们将引入在文献中经常出现的一类非常有用的Lyapunov函数,再通过状态方程的线性化反过来研究非线性系统的稳定性,并了解此法有何局限性。

  在研究LTI系统 \eqref{LTI} 时,常用下面的二次型作为候选Lyapunov函数
  \[ V (x) = x^\mathrm{T} P  x \]
  其中$P \in \mathbb{R}^{n \times n}$ 为正定。求其导数
  \begin{align*}
    \dot{V}(x)&= x^\mathrm{T}P\dot{x}+\dot{x}^\mathrm{T}Px\\
    &=x^\mathrm{T}PAx+(Ax)^\mathrm{T}Px\\
    &=x^\mathrm{T} (P  A + A^\mathrm{T} P) x \\
    &\triangleq - x^\mathrm{T} Q  x
  \end{align*}
  其中$Q$是定义为
  \begin{equation}
    P  A + A^\mathrm{T} P = - Q \label{lyaLTI}
  \end{equation}
  的对称矩阵。
  若$Q$正定,那么系统 \eqref{LTI} 的原点就是渐近稳定的。

  下面给出LTI系统的稳定性判据。此定理的证明可参看《线性系统理论》相关教材。
  
  \begin{theorem}[线性定常系统的稳定性判据]\label{linearlya}
    系统 \eqref{LTI} 中的原点是渐近稳定的,或$A$是Hurwitz的(意即其所有特征值均具有负实部),当且仅当
    对于任意给定的对称正定矩阵$Q$,都存在唯一的对称正定矩阵$P$使得 \eqref{lyaLTI} 成立。
  \end{theorem}
  \begin{note}
    注意,此判据是充要条件。
  \end{note}
  
  下面利用线性化,将这个结论推广至非线性系统。

  以分析原点的稳定性为例。将系统 \eqref{freeofauto} 在原点附近线性化得
  \begin{equation*}
    \dot{x}=Ax+g(x)
  \end{equation*}
  其中
  \begin{equation}\label{jacobian}
    A=\frac{\partial f(x)}{\partial x}|_{x=0}=\left.\begin{bmatrix}\frac{\partial f_{1}}{\partial x_{1}}&\cdots&\frac{\partial f_{n}}{\partial x_{1}}\\\vdots&&\vdots\\\frac{\partial f_{n}}{\partial x_{1}}&\cdots&\frac{\partial f_{n}}{\partial x_{n}}\end{bmatrix}\right|_{x=0}
  \end{equation}
  称为雅可比矩阵(Jacobian)\index{雅可比矩阵(Jacobian)},且有(所用范数均为2-范数)
  \begin{equation}\label{gxto0}
    \|x\|\to0\ \text{时}\ \cfrac{\|g(x)\|}{\|x\|}\to0
  \end{equation}
  考虑候选Lyapunov函数
  \[ V (x) = x^\mathrm{T} P  x \]
  其中$P \in \mathbb{R}^{n \times n}$ 为正定。其导数等于
  \begin{equation*}
    \begin{aligned}
    \dot{V}&= x^\mathrm{T}P\dot{x}+\dot{x}^\mathrm{T}Px\\
    &= x^\mathrm{T}P(Ax+g(x))+(Ax+g(x))^\mathrm{T}Px\\
    &= x^\mathrm{T}(PA+A^\mathrm{T}P)x+x^\mathrm{T}Pg(x)+g^\mathrm{T}(x)Px\\
    &=-x^\mathrm{T}Qx+2x^\mathrm{T}Pg(x)
    \end{aligned}
  \end{equation*}
  其中第四个等号,是由于$x^\mathrm{T}Pg(x)$为标量,所以转置与本身相等,
  而其转置刚好就是最后一项(利用$P$的对称性)。
  
  因为$A$是Hurwitz的,根据 \ref{linearlya} 知$Q$正定;此外,由 \eqref{gxto0},
  对$\forall \gamma>0$,$\exists r>0$使得\[\frac{\|g(x)\|}{\|x\|}\le \gamma,\quad\forall\|x\|\le r\]
  因此$\|x\|\le r$时有\begin{align*}
    \dot{V}&\le -x^\mathrm{T}Qx+2\|x\|\|P\|\|g(x)\|\\
    &\le -x^\mathrm{T}Qx+2\gamma\|x\|^2\|P\|\\
    &\le -\lambda_{\min}(Q)\|x\|^2+2\gamma \|P\|\|x\|^2\\
    &= -(\lambda_{\min}(Q)-2\gamma\|P\|)\|x\|^2
  \end{align*}
  第一个不等号是由于范数相容性,第三个不等号是由不等式 \ref{rayleigh-ritz}。
  这样,选$\gamma$使得$\lambda_{\min}(Q)>2\gamma \|P\|$,即可使得$\dot{V}$是负定的。此时$x=0$就是渐近稳定的。

  将上述方法归纳为下面定理。
  \begin{theorem}[通过线性化判别非线性系统稳定性]
    对于系统 \eqref{freeofauto},如果对于式 \eqref{jacobian} 所示雅可比矩阵$A$\index{雅可比矩阵(Jacobian)},有
    \begin{itemize}[leftmargin=1em]
    \item $\forall i, \ensuremath{\operatorname{Re}} [\lambda_i (A)] < 0$,则系统 \eqref{freeofauto} 的原点是渐近稳定的;
    \item $\exists i, \ensuremath{\operatorname{Re}} [\lambda_i (A)] > 0$,则系统 \eqref{freeofauto} 的原点是不稳定的;
    \item $\forall i, \ensuremath{\operatorname{Re}} [\lambda_i (A)] \le 0$,且
    $\exists j, \ensuremath{\operatorname{Re}} [\lambda_j (A)] = 0$,无法判断。
    \end{itemize}
  \end{theorem}
  \begin{note}
    上面定理的直观解释:在平衡点附近线性化,也可看作对所得线性系统施加小扰动。对于所得线性系统特征值严格落在左半平面或右半平面的情形,
    这种扰动不会使得特征值跨越虚轴从而改变系统性质;然而,若所得线性系统存在虚轴上的特征值,则扰动可能使其进入左半平面,也可能
    使其进入右半平面,因此我们无法做出判断。

    由于线性化过程依赖于“平衡点附近”的假设,因此只能得出局部渐近稳定的结论,这也就是此法的局限性。
  \end{note}
  \newpage
  \begin{example}[通过平衡点附近的线性化判别单摆系统的稳定性]\label{linearization_exp}
   考虑单摆系统 \eqref{damped}。利用线性化检验$(0, 0)$ 和 $(\pi, 0)$这两个平衡点的稳定性。
    \[ A_1 = \left. \frac{\partial f}{\partial x} \right|_{x = (0, 0)} = \left.
       \left[\begin{array}{cc}
         0 & 1\\
         -a\ensuremath{\operatorname{cos}}  x_1 & - b
       \end{array}\right] \right|_{x = (0, 0)} = \left[\begin{array}{cc}
         0 & 1\\
         - a & - b
       \end{array}\right] \]
    特征方程是 $\lambda^2 + b \lambda + a = 0$,两特征值之和为负、之积为正,则一定都在左半平面。因此该系统在$(0, 0)$附近是稳定的。
    \[ A_2 = \left. \frac{\partial f}{\partial x} \right|_{x = (\pi, 0)} =
       \left. \left[\begin{array}{cc}
         0 & 1\\
         -a\ensuremath{\operatorname{cos}}  x_1 & - b
       \end{array}\right] \right|_{x = (\pi, 0)} = \left[\begin{array}{cc}
         0 & 1\\
         a & - b
       \end{array}\right] \]
       特征方程是 $\lambda^2 + b \lambda - a = 0$,两特征值之积为负,则一定有处于右半平面的特征值。因此该系统在$(\pi, 0)$附近不稳定。
      从物理上考虑则不难理解:$(\pi, 0)$对应的实为单摆竖立最高点(最低点的对称点)。
    \begin{note}
      对于一元二次方程 $a  x^2 + b  x + c = 0$,其解$x_{1},x_{2}$满足韦达定理
      \[ x_1 + x_2 = \frac{- b}{a}, \quad x_1  x_2 = \frac{c}{a} \]
      对于$a = 1$情形,若$c > 0$则两根同号;若还有$b>0$,那么两根均具有负实部。
      同样对于$a = 1$情形,若$c < 0$则两根异号。
    \end{note}
  \end{example}