\section{不变集原理}\label{2Dref}
我们总更希望一个系统是渐近稳定而不仅仅是稳定的。然而,如例 \ref{suff_lyapunov} 所示,我们构造的Lyapunov函数之导数常常只能是负半定而非负定的,从而无法得出渐近稳定的结论。下面将给出Lyapunov稳定性定理的一个推广,它有助于解决此问题。首先介绍相关概念。
\begin{definition}[不变集(invariant set)]\index{不变集(invariant set)}
	称集合$M$是关于式 \eqref{freeofauto} 的{\bf 正不变集(positively invariant set)},
	若\[x(0)\in M\implies x(t)\in M,\forall t\ge0\]
	换言之,每个自$M$内出发的状态轨线,必整体留在$M$内。
\end{definition}
\begin{definition}[趋于集合(approach a set)]
	称$t\to\infty$时,$x(t)$趋于一集合$M$,若\[\forall \varepsilon>0,\exists T>0:\operatorname{dist}(x(t),M)<\varepsilon,\forall t>T\]
	其中$\operatorname{dist}(p,M)=\inf\limits_{x\in M}\|p-x\|$。换言之,时间足够长时,$x(t)$与$M$的“距离最小值”将趋于0。
\end{definition}
\begin{example}[不变集]\label{invset}
	\begin{itemize}[leftmargin=1em]
		\item 平衡点
		\item $\R{}^n$
		\item $\Omega_c=\{x\in\R{}^n:V(x)\le c\}$,其中$\dot{V}(x)\le 0$。此不变集之后很常用到。
		\item 极限环\index{极限环(limit cycle)}:称一状态轨线是封闭的,若其非常值,且存在$t_0>0$,使得$x(t+t_0)=x(t)$(总会复归之前的状态)。称一孤立的封闭状态轨线为极限环。例子可见 \ref{stable_unstable}。其类别有
		\begin{itemize}[leftmargin=1em]
			\item 稳定极限环:极限环附近的所有轨迹皆收敛于它($t\to\infty$)
			\item 不稳定极限环:极限环附近的所有轨迹皆不收敛于它($t\to\infty$)
			\item 半稳定极限环:极限环附近的轨迹,部分收敛于它,另一部分发散($t\to\infty$)
		\end{itemize}
	\end{itemize}
\end{example}
有了上面的铺垫,我们不加证明地给出下述定理。
\begin{theorem}[LaSalle不变集原理]\label{lasalle}\index{LaSalle不变集原理}
	对于系统 \eqref{freeofauto},设\begin{itemize}[leftmargin=1em]
		\item $\Omega\subset D$为紧的正不变集;
		\item $V:D\to\R{}$是连续可微函数,且在$\Omega$中满足$\dot{V}(x)\le 0$;
		\item $E=\{x\in\Omega: \dot{V}(x)= 0\}$;
		\item $M$为$E$中最大的不变集。
	\end{itemize}
	那么当$t\to\infty$,每个自$\Omega$中出发的轨线都将趋于$M$。
\end{theorem}
\begin{note}
	不变集可视为平衡点的推广。注意,$V$并不需要是正定函数。
\end{note}
\begin{corollary}
	令$x=0$为 \eqref{freeofauto} 的平衡点。$D\in\R{}^n$是包含原点的域。令$V:D\to\R{}$是$C^1$正定函数,且$\dot{V}\le 0$在$D$上总成立。
	令$E=\{x\in D:\dot{V}=0\}$,并设除了平凡解$x(t)=0$外,没有解能一直处于$E$中。那么$x=0$是渐近稳定的。
\end{corollary}
\begin{corollary}\label{lasallegas}
	令$x=0$为 \eqref{freeofauto} 的平衡点。令$V:\R{}^n\to\R{}$是径向无界的$C^1$正定函数,且$\dot{V}\le 0$在$\R{}^n$上总成立。
	令$E=\{x\in \R{}^n:\dot{V}=0\}$,并设除了平凡解$x(t)=0$外,没有解能一直处于$E$中。那么$x=0$是全局渐近稳定的。
\end{corollary}
\begin{note}
	在$\dot{V}$负定时,$E=\{0\}$,上述推论即退化为 \ref{lyapunov}。
\end{note}
\begin{example}[利用LaSalle定理判别单摆系统稳定性]
	考虑例 \ref{suff_lyapunov}。首先考虑无阻尼的情形 \eqref{nodamp},Lyapunov函数\[V(x)=\frac{1}{2}x_2^2+a(1-\cos x_1)\]
	可构造不变集$\Omega$类似例 \ref{invset} 第三点(不过需注意定义域):\[\Omega=\{x:V(x)\le c\},c>0\]
	例 \ref{suff_lyapunov} 已经计算过$\dot{V}=0$,因此$E=\Omega$。
	$E$中最大的不变集$M$,显然就是$\Omega$本身。于是利用 \ref{lasalle},$\Omega$中出发的轨线都将趋于$\Omega$。

	再考虑有阻尼的情形 \eqref{damped}。仍然选择$\Omega$如上。由于$\dot{V}=0\iff x_2=0$,因此\[E=\{x:x_2=0,x\in\Omega\}\]
	再找$E$中最大的不变集。由于自$x_1\ne 0,x_2=0$起始的状态轨线,一经出发,必然使得$x_2\ne 0$(从物理意义上考虑:在非最低点从静止释放单摆,
	单摆必定马上具有非零速度)。因此$M=\{(0,0)\}$。此时利用 \ref{lasalle},可得每个自$\Omega$中出发的轨线都将趋于原点。我们发现,这样也说明了原点的渐近稳定性。
\end{example}
\begin{example}[利用LaSalle定理判别自适应控制的收敛性]
	考虑如下一阶系统
  \[ \dot{y} = a  y + u \]
  对其作用自适应控制律
  \[ u = - k  y, \dot{k} = \gamma y^2, \gamma > 0. \]
  取$x_1 = y, x_2 = k$,我们可得
  \[ \left\{\begin{array}{l}
       \dot{x}_1 = a  x_1 - x_1 x_2\\
       \dot{x}_2 = \gamma x^2_1
     \end{array}\right. \]
  $\{x:x_1 = 0\}$是一个由平衡点构成的集合(将$x_1=0$代入验证即可)。我们想要说明在$t \rightarrow \infty$
  时状态轨线将收敛到这个集合上,从而说明自适应律可使$y$收敛至$0$。
  
  考虑如下函数
  \begin{equation}\label{adaptive_eg_candidate}
	V (x) = \frac{1}{2} x^2_1 + \frac{1}{2} (x_2 - k_0)^2
  \end{equation}
  其中$k_0$为常数,加入此项的缘由是上述平衡点集并没有保证$x_2=0$。注意,这个函数并不是正定的
  (因其在原点取值非零),但是这并不违背LaSalle定理的条件。
  
  $V$随时间的导数为
  \begin{align*}
    \dot{V} & =  x_1 \dot{x}_1 +  (x_2 - k_0) \dot{x}_2\\
    & =  x_1 (a  x_1 - x_1 x_2) + (x_2 - k_0) \gamma x^2_1\\
	&= ax_1^2-x_1^2x_2+\gamma(x_2x_1^2-k_0x_1^2)
  \end{align*}
  我们发现,如果在 \eqref{adaptive_eg_candidate} 第二项中加入$\frac{1}{\gamma}$,那么上式就有可消去的项。则
  \[V (x) = \frac{1}{2} x^2_1 + \frac{1}{2\textcolor{structurecolor}{\gamma}} (x_2 - k_0)^2, \dot{V}  =  (a - k_0)  x^2_1\]
  取 $k_0 > a$,则 $\dot{V} \leq 0$。

  接下来应用LaSalle定理。首先定义不变集$\Omega = \{ x \in \mathbf{R}^2 : V (x) \leq c \},c>0$。
  则集合$E =\{ x \in \Omega :\dot{V}= 0 \}= \{ x \in \Omega : x_1 = 0 \}$。进而由本例开头的分析知$M = E$。
  根据 \ref{lasalle},解会收敛至$x_1=0$上,也即$\lim\limits_{t \rightarrow \infty} x_1 (t) = 0$。
  又知$V(x)$径向无界,所以由 \ref{lasallegas},此结论全局成立。
\end{example}
定理 \ref{lasalle} 除了可用于判别趋于原点的情形,还可推广用于判别收敛至极限环等。
\begin{example}[收敛至极限环]
	考虑下述非线性系统\begin{align*}
		\dot{x}_1&=x_2+x_1(\beta^2-x_1^2-x_2^2)\\
		\dot{x}_2&=-x_1+x_2(\beta^2-x_1^2-x_2^2)
	\end{align*}
	原点是平衡点(通过验证$x_1=x_2=0$时$\dot{x}_1=\dot{x}_2=0$即可)。

	我们又发现,如果$x_1^2+x_2^2=\beta^2$,那么系统方程改写为\[\left\{\begin{array}{l}
    \dot{x}_1 = x_2\\
    \dot{x}_2 = - x_1
  \end{array}\right.\]
  对$x_1^2+x_2^2=\beta^2$求导并将上面的式子代入可发现恒成立(也可通过几何层面,由上述方程也表示圆的方程看出),
  这样一来,自$x_1^2+x_2^2=\beta^2$出发的轨线会一直在其上运动,因此形成了一个极限环,同时也是不变集。

  \begin{note}
    定性分析极限环性质:设$r = \sqrt{x^2_1 + x^2_2}, \theta = \arctan \frac{x_2}{x_1}$,我们有(注意此时应代入原方程)
  \begin{align*}
    \dot{r}& = \frac{1}{2}(x_{1}^{2}+x_{2}^{2})^{-\frac{1}{2}}(2x_{1}\dot{x}_{1}+2x_{2}\dot{x}_{2})= r (\beta^2 - r^2)\\
    \dot{\theta} & =\frac{1}{1+(\frac{x_{2}}{x_{1}})^{2}}\frac{\dot{x}_{2}x_{1}-\dot{x}_{1}x_{2}}{x_{1}^{2}}= - 1
  \end{align*}
  其中用到了
  \begin{align*}
    x_1\dot{x}_1+x_2\dot{x}_2& = (x_1^2+x_2^2)(\beta^2-x_1^2-x_2^2)= r^2 (\beta^2 - r^2)\\
    x_1\dot{x}_2-x_2\dot{x}_1 & = -(x_1^2+x_2^2)=- r^2
  \end{align*}
  则 \ $\left\{\begin{array}{l}
    r < \beta, \dot{r} > 0 \Rightarrow r \rightarrow \beta\\
    r > \beta, \dot{r} < 0 \Rightarrow r \rightarrow \beta
  \end{array}\right. \Rightarrow$ 极限环稳定。
\end{note}

  下面利用LaSalle定理来分析。构造如下候选Lyapunov函数\[ V (x) = \frac{1}{4} (x^2_1 + x^2_2 - \beta^2)^2 \]
  构造此函数的动机是,它衡量了与极限环的“距离”。显然$V (x) \geq 0$,但要注意同上个例子一样,此函数也不是正定的。求其导数得
  \begin{align*}
    \dot{V} (x) & =  \frac{1}{2} (x^2_1 + x^2_2 - \beta^2) (2 x_1 \dot{x}_1 +
    2 x_2 \dot{x}_2)\\
    & =  - (x^2_1 + x^2_2 - \beta^2)^2 (x^2_1 + x^2_2)\leq 0
  \end{align*}
  构造集合$\Omega_c =
  \{ x \in \mathbf{R}^2 : V (x) \leq c \},c > 0$。因为$\dot{V} \leq 0$,所以这是不变集。
  接下来寻求$E = \{ x \in \Omega_c : \dot{V} (x) = 0 \}$。根据导函数形式可得$E = \{
  (0, 0) \} \cup \{ x : x^2_1 + x^2_2 = \beta^2 \}$。
  不过,要注意这只在$c \ge \frac{1}{4} \beta^4$时成立——因为$0 < c < \frac{1}{4} \beta^4$时,
  $\Omega_c $不包含原点,此时$E=\{ x : x^2_1 + x^2_2 = \beta^2 \}$。
  最后,根据前面所做的分析,$E$中的最大不变集$M = E$。
  
  这样,利用 \ref{lasalle},其中$c \in \left( 0,
  \frac{1}{4} \beta^4 \right)$,即知每个自$\Omega_c$出发的轨线均收敛至极限环,因此极限环稳定。
  取$c=\frac{1}{4} \beta^4-\varepsilon$,其中$\varepsilon$可任意小,我们发现,无限接近于原点的状态也都会收敛至极限环,则原点不稳定。

  \begin{note}
    要判别原点的不稳定性,亦可构造如下候选Lyapunov函数\[V=\frac12(x_1^2+x_2^2)\]
  求导得\begin{align*}
	  \dot{V}&=x_1\dot{x}_1+x_2\dot{x}_2=(x_1^2+x_2^2)(\beta^2-x_1^2-x_2^2)
  \end{align*}
  其在原点的小邻域内为正定,则运用定理 \ref{unstable} 即可。
  \end{note} 
\end{example}
