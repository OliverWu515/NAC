\section{比较函数}\label{3Aref}

\begin{definition}[$\mathcal{K}$ 类函数、 $\mathcal{K}_{\infty}$类函数]
  称连续函数 $\alpha : [0, a) \rightarrow [0, \infty)$ 属于 $\mathcal{K}$ 类\index{K@$\mathcal{K}$ 类函数},若其严格单调递增且 $\alpha (0) =0$。
  进一步地,称其属于 $\mathcal{K}_{\infty}$类\index{Kinf@$\mathcal{K}_{\infty}$ 类函数},若 $a = \infty$ 且$r \rightarrow \infty$时
  $\alpha (r) \rightarrow \infty$。
\end{definition}

\begin{definition}[$\mathcal{K}\mathcal{L}$类函数]
  称连续函数 $\beta : [0, a) \times [0, \infty) \rightarrow [0,
  \infty)$ 属于 $\mathcal{K}\mathcal{L}$ 类\index{Kl@$\mathcal{K}\mathcal{L}$ 类函数},若映射$\beta (r, t)$对每个取定的$t$,关于$r$ 属于 $\mathcal{K}$类,
  并且对每个取定的$r$,随着$t
  \rightarrow \infty$,$\beta (r, t)$单调递减并趋于$ 0$。
\end{definition}

\begin{example}  [$\mathcal{K}$ 类、 $\mathcal{K}_{\infty}$类、$\mathcal{K}\mathcal{L}$类函数]
  \begin{itemize}[leftmargin=1em]
    \item $\alpha (r) = \arctan (r)$。由$\alpha' (r) = \frac{1}{r^2 + 1} \geq 0,
    \alpha (0) = 0, \lim\limits_{r \rightarrow \infty} \alpha (r) = \frac{\pi}{2}$,
    所以$\alpha (r)$ 属于 $\mathcal{K}$类函数,但不属于 $\mathcal{K}_{\infty}$类函数。
    
    \item $\alpha (r) = r^c$。由$ \alpha' (r) = c r^{c-1} \geq 0, \alpha (0) = 0, \lim\limits_{r
    \rightarrow \infty} \alpha (r) = \infty$,所以 $\alpha (r)$ 属于 $\mathcal{K}_{\infty}$类函数。
    
    \item $\alpha (r) = \min \{ r, r^2 \}$ 属于 $\mathcal{K}_{\infty}$类函数。
    
    \item $\beta (r, t) = r^2 \mathrm{e}^{- t}$ 属于    $\mathcal{K}\mathcal{L}$类函数。
    \item $\beta (r, t) = \frac{r}{rt+1}=\frac{1}{t+\frac1r}$ 属于 $\mathcal{K}\mathcal{L}$类函数。
  \end{itemize}
\end{example}

\begin{lemma}[$\mathcal{K}$类、$\mathcal{K}_\infty$类、$\mathcal{K}\mathcal{L}$类函数的性质]
  设定义在$[0,a)$上的函数$\alpha_{1},\alpha_{2}\in\mathcal{K}$,$\alpha_{3},\alpha_{4}\in\mathcal{K}_{\infty}$,且有
  $\beta\in\mathcal{KL}$。令$\alpha_{i}^{-1}$表示$\alpha_{i}$的反函数。则
  \begin{itemize}[leftmargin=1em]
    \item $\alpha_{1}^{-1}$定义在$[0,\alpha_{1}(a))$上,属于$\mathcal{K}$类函数;
    \item $\alpha_{3}^{-1}$定义在$[0,\infty)$上,属于$\mathcal{K}_\infty$类函数;
    \item $\alpha_{1} \circ\alpha_{2}$属于$\mathcal{K}$类函数;
    \item $\alpha_{3} \circ\alpha_{4}$属于$\mathcal{K}_\infty$类函数;
    \item $\sigma(r,t)=\alpha_{1}(\beta(\alpha_{2}(r),t)) $属于$\mathcal{K}\mathcal{L}$类函数。
  \end{itemize}
\end{lemma}

通过下面的引理,我们将$\mathcal{K}$和$\mathcal{K}_\infty$类函数引入Lyapunov分析中。

\begin{lemma}\label{lyapunov_comp}
 设 $V : D \rightarrow \mathbb{R}$ 是连续的正定函数,$D\subset\mathbb{R}^n$是包含原点的域。
  令$B_r\subset D$($r>0$)。那么存在定义在$[0,r]$上的
  $\mathcal{K}$ 类函数$\alpha_1,\alpha_2$使得
  \[ \alpha_1 (\| x \|) \leq V (x) \leq \alpha_2 (\| x \|), \forall x \in B_r \]
  若$D =\mathbb{R}^n$,那么$\alpha_1,\alpha_2$定义在$[0,\infty)$上。
  进一步,若$V(x)$是径向无界的,那么 $\alpha_1$与 $\alpha_2$可选为 $\mathcal{K}_{\infty}$类函数。
\end{lemma}

\begin{example}
  设$V (x) = x^\mathrm{T} P  x$,$P$正定。由 \ref{rayleigh-ritz},可取上述引理中的$\alpha_1(r)=\lambda_{\min} (P)r^2$,$\alpha_2(r)=\lambda_{\max} (P)r^2$。
\end{example}
为了看出 \ref{lyapunov_comp} 是如何用于Lyapunov分析的,下面利用它来证明定理 \ref{lyapunov}。
在证明中,我们尝试选取$\delta$和$\beta$使得$B_\delta\subset\Omega_\beta\subset B_r$。利用上述引理
我们知道\[\alpha_1 (\| x \|) \leq V (x) \leq \alpha_2 (\| x \|)\]
可取$\beta\le\alpha_1(r)$和$\delta\le\alpha_2^{-1}(\beta)$(即$\alpha_2(\delta)\le\beta$)。首先可得
\[V(x)\le\beta\implies\alpha_1(\|x\|)\le\alpha_1(r)\iff \|x\|\le r\]
其中的等价号是由$\alpha_1$的单调性。这样就得到了$\Omega_\beta\subset B_r$。
还可得\[\|x\|\le\delta\implies V(x)\le\alpha_2(\delta)\le\beta\]
这样就得到了$B_\delta\subset\Omega_\beta$。
关于渐近稳定性的证明,尚需要一些其他的结论,可参考前言中的参考文献1(145-146页)。

比较函数的更多应用见下一节。

% 下面两个引理也是利用比较函数进行Lyapunov分析的例子,可自行证明之。
% \begin{lemma}
%   系统 (\ref{freeofauto}) 的平衡点 $x = 0$ 是稳定的,当且仅当存在一$\mathcal{K}$类函数 $\alpha (\cdot)$
%   和常数$\delta$ 使得
%   \[ \| x (0) \| \leq \delta \quad \Rightarrow \quad \| x (t) \| \leq \alpha
%      (\| x (0) \|), \forall t \geq 0. \]
% \end{lemma}

% \begin{lemma}
%   系统 (\ref{freeofauto}) 的平衡点 $x = 0$ 是渐近稳定的,当且仅当存在一
%   $\mathcal{K}\mathcal{L}$类函数 $\beta (\cdot,\cdot)$与常数 $\delta$
%   使得
%   \[ \| x (0) \| \leq \delta \quad \Rightarrow \quad \| x (t) \| \leq \beta
%      (\| x (0) \|, t), \forall t \geq 0. \]
% \end{lemma}