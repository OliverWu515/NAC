\section{线性时变系统}\label{3Dref}
在进一步研究非自治系统稳定性之前,我们先简短讨论如下线性时变系统。同样地,其可看成非自治的非线性系统的一个特例。
\begin{equation}
  \dot{x}(t) = A (t) x(t) \label{Sys:LTVS}
\end{equation}
方程 \eqref{Sys:LTVS} 的解具有形式
\[ x (t) = \Phi (t, t_0) x (t_0) \]
其中$\Phi (t, t_0)$是状态转移矩阵。

\begin{theorem}
 系统 \eqref{Sys:LTVS} 的平衡点$x = 0$ 是一致指数稳定的,当且仅当状态转移矩阵满足下面不等式
  \[ \| \Phi (t, t_0) \| \leq k  \mathrm{e}^{- \lambda (t - t_0)}, \forall t \geq t_0
     \geq 0 \]
  其中 $k$ 和 $\lambda$为正常数。
\end{theorem}
\begin{note}
  对于线性系统而言,(全局)一致渐近稳定 $\iff$ (全局)一致指数稳定。
\end{note}
不同于线性定常系统,对于线性时变系统,并不能用$A (t)$特征根的位置来判别稳定性。下面用一个例子来说明。
\begin{example}[特征根具有负实部,但原点不稳定]
  考虑具有下述系统矩阵的二阶线性系统
  \[ A (t) = \left[\begin{array}{cc}
       - 1 + 1.5 \cos^2 t & 1 - 1.5 \sin  t \cdot \cos  t\\
       - 1 - 1.5 \sin  t \cdot \cos  t & - 1 + 1.5 \sin^2 t
     \end{array}\right] \]
 特征方程 $| \lambda I - A | = \lambda^2 + 0.5 \lambda + 0.5 = 0$,
 由例 \ref{linearization_exp} 的注记可知其两个特征根都具有负实部。
  
  然而 $\Phi (t, 0) = \left[\begin{array}{cc}
    \mathrm{e}^{0.5 t} \cos  t & \mathrm{e}^{- t} \sin  t\\
    -\mathrm{e}^{0.5 t} \sin  t & \mathrm{e}^{- t} \cos  t
  \end{array}\right]$,其第一列元素是发散的。
\end{example}

\begin{corollary}\label{ltvs_lambda}
  系统 \eqref{Sys:LTVS} 是一致指数稳定的,若 $\exists \lambda > 0$ 使得
  \[ \lambda_i (A (t) + A^\mathrm{T} (t)) \leq - \lambda, \forall t \geq 0, i = 1,
     \ldots, n \]
\end{corollary}
\begin{proof}
    设 $V = x^\mathrm{T} x$,显然$V$正定。并且
  \begin{align*}
    \dot{V} & =  x^\mathrm{T} (A (t) + A^\mathrm{T} (t)) x\\
    & \leq  - \lambda x^\mathrm{T} x\\
    & =  - \lambda V
  \end{align*}
  利用 \ref{lyapunov_nonauto} 第三条即可。
\end{proof}

\begin{corollary}
  假设存在连续可微且对称的 $P (t) \in \mathbf{R}^{n \times n}$,并存在 $0 < c_1 < c_2 < \infty$
  使得
  \begin{equation}\label{pbounded}
    c_1 I \leq P (t) \leq c_2 I, \forall t \geq 0
  \end{equation}
  进一步,假设某个连续可微且对称的 $Q (t) \in \mathbf{R}^{n \times n}$满足对于任意$t>0$有
  \[ Q (t) \geq c_3 I > 0 \]
  \[ \dot{P} (t) + P (t) A (t) + A^\mathrm{T} (t) P (t) = - Q (t) \]
  那么 \eqref{Sys:LTVS} 的原点是全局一致渐近稳定(也是全局一致指数稳定)的。
\end{corollary}
\begin{proof}
    令 $V (x, t) = x^\mathrm{T} P (t) x$。利用 \eqref{pbounded} 可得
  $W_1 (x) \triangleq c_1 x^\mathrm{T} x \leq V (x, t) \leq c_2 x^\mathrm{T} x \triangleq W_2 (x)$,
  即$V$正定且渐弱。
  
  求$V$的导数可得
  \begin{align*}
    \dot{V} & =  x^\mathrm{T} P (t) A (t) x + x^\mathrm{T} A^\mathrm{T} (t) P (t) x + x^\mathrm{T} \dot{P} (t) x\\
    & =  - x^\mathrm{T} Q (t) x\\
    & \leq  - c_3 x^\mathrm{T} x
  \end{align*}
  定义$W_3 (x)\triangleq x^\mathrm{T} x $,利用 \ref{lyapunov_nonauto} 第四、五条即可。
\end{proof}

\begin{corollary}
  假设对于任意时刻 $t \geq 0$, $A (t)$ 的所有特征值都具有负实部,并且$A (t)$有界且 $\int^{\infty}_0
  A^\mathrm{T} (t) A (t) \diff  t < \infty$,那么系统 \eqref{Sys:LTVS} 的平衡点$x=0$是全局一致渐近稳定的。
\end{corollary}
\begin{theorem}[利用线性化判别非自治系统的稳定性]
    令 $x = 0$ 是系统 \eqref{freeofnonauto} 的平衡点,其中$f$是$C^1$的,$D=\{x\in \mathbf{R}^n:\|x\|<r\}$。
    雅可比矩阵$\frac{\partial f}{\partial x}$有界且在$D$上Lipschitz(对$t$一致)。令$A(t)=\left.\frac{\partial f}{\partial x}(t,x)\right|_{x=0}$。
    那么原点是指数稳定的,当且仅当其对于线性系统 \eqref{Sys:LTVS}(以上述$A(t)$为系统矩阵)是指数稳定的。
\end{theorem}