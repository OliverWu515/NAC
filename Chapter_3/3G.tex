\section{输入-状态稳定性}\label{3Gref}

考虑系统
\begin{equation} \label{sys:I:ss}
  \dot{x} = f (t, x, u) \text{,}
\end{equation}
其中 $f : (0, \infty) \times \mathbb{R}^n \times \mathbb{R}^n \to \mathbb{R}^n$ 是关于 $t$ 分段连续且关于 $x$ 和 $u$ 局部 Lipschitz 的函数。输入 $u$ 是 $t$ 的有界连续函数。假设无外力系统
\begin{equation}
    \dot{x} = f (t, x, 0)
\end{equation}
具有全局一致渐进稳定的平衡点 $x = 0$。那么,在有界输入 $u(t)$ 下,系统 \eqref{sys:I:ss} 会有怎样的表现?首先我们从简单情形出发,讨论线性时不变系统。

\begin{example}
    对于线性时不变系统
    \[
        \dot{x} = A  x + B  u \text{,}
    \]
    其中 $A$ 是 Hurwitz 的,我们可以得到其解为
    \[
        x (t) = \mathrm{e}^{A (t - t_0)} x (t_0) + \int^t_{t_0} e^{A (t - \tau)} B  u (\tau) \diff \tau \text{。}
    \]
    利用 $\| \mathrm{e}^{A (t - t_0)} \| \leq k  \mathrm{e}^{- \lambda (t - t_0)}$(其中 $k, \lambda \geq 0$),可以对这个解做如下近似:
    \begin{eqnarray*}
        \| x (t) \| & \leq & k  \mathrm{e}^{- \lambda (t - t_0)} \| x (t_0) \| + k
        \int^t_{t_0} \mathrm{e}^{- \lambda (t - \tau)} \| B \| \|  u (\tau) \| \diff \tau\\
        & \leq & k  \mathrm{e}^{- \lambda (t - t_0)} \| x (t_0) \| + k \| B \| \sup_{t_0
        \leq \tau \leq t} \|  u (\tau) \| \int^t_{t_0} \mathrm{e}^{- \lambda (t - \tau)} \diff \tau\\
        & \leq & k  \mathrm{e}^{- \lambda (t - t_0)} \| x (t_0) \| + \frac{k \| B \|}{\lambda} \sup_{t_0 \leq \tau \leq t} \|  u (\tau) \| \text{。}
    \end{eqnarray*}
    由此可知,解当中的零输入响应(上式最后一行的第一项)按指数速率衰减至零,而零状态响应(上式最后一行的第二项)对于每个有界的输入都是有界的。也就是说,对于任意有界的输入,系统的状态都有界。
\end{example}

接下来,我们讨论非线性系统在什么情况下具有这样的性质。对应的无外力系统具有全局一致渐进稳定平衡点即可吗?答案是否定的。

%%%

% \begin{example}
%   For the linear time-invariant system
%   \[ \dot{x} = A  x + B  u \]
%   with a Hurwitz matrix $A$. Its solution is
%   \[ x (t) = e^{A (t - t_0)} x (t_0) + \int^t_{t_0} e^{A (t - \tau)} B  u
%      (\tau) d \tau \]
% \end{example}

% We can use $\| e^{A (t - t_0)} \| \leq k  e^{- \lambda (t - t_0)}$ with $k,
% \lambda \geq 0$, to estimate the solution
% \begin{eqnarray*}
%   \| x (t) \| & \leq & k  e^{- \lambda (t - t_0)} \| x (t_0) \| + k
%   \int^t_{t_0} e^{- \lambda (t - \tau)} \| B \| \|  u (\tau) \| d \tau\\
%   & \leq & k  e^{- \lambda (t - t_0)} \| x (t_0) \| + k \| B \| \sup_{t_0
%   \leq \tau \leq t} \|  u (\tau) \| \int^t_{t_0} e^{- \lambda (t - \tau)} d
%   \tau\\
%   & \leq & k  e^{- \lambda (t - t_0)} \| x (t_0) \| + \frac{k \| B
%   \|}{\lambda} \sup_{t_0 \leq \tau \leq t} \|  u (\tau) \|
% \end{eqnarray*}
% The zero-input response decays to zero exponentially fast, while the state is
% bounded for every bounded input.

% \begin{definition}
%   The system \eqref{sys:I:ss} is said to be {\textbf{input-to-state stable}}
%   (ISS) if there exist a class $\mathcal{K}\mathcal{L}$ function $\beta$ and a
%   class $\mathcal{K} $ function $\gamma$ such that for any initial state $x
%   (t_0)$ and any bounded input $u (t)$, the solution $x (t)$ exists for all $t
%   \geq t_0$ and satisfies
%   \[ \| x (t) \| \leq \beta (\| x (t_0) \|, t - t_0) + \gamma (\sup_{t_0 \leq
%      \tau \leq t} \|  u (\tau) \|) \]
% \end{definition}

% \begin{note}
%   {$\mbox{}$}
  
%   \begin{enumerate}
%     \item For any bounded $u (t)$, $x (t)$ will be bounded.
    
%     \item as $t$ increases, $x (t)$ will be \text{{\ttfamily{ultimately
%     bounded}}} by a class $\mathcal{K}$ function of $\sup_{t_0 \leq \tau \leq
%     t} \|  u (\tau) \| .$
    
%     \item If $u (t) \equiv 0$, then $x = 0$ is asymptotically stable.
    
%     \item If $u (t) \rightarrow 0$ as $t \rightarrow \infty$, so does $x (t)$.
%   \end{enumerate}
% \end{note}

% \begin{lemma}
%   If the unforced system $\dot{x} = f (t, x, 0)$ has a globally exponentially
%   stable equilibrium point at $x = 0$, then the system $\dot{x} = f (t, x, u)$
%   is ISS.
% \end{lemma}

% \begin{example}
%   An interesting application. The cascade system
%   \begin{eqnarray}
%     \dot{x}_1 & = & f_1 (t, x_1, x_2)  \label{cascade:state:1}\\
%     \dot{x}_2 & = & f_2 (t, x_2)  \label{cascade:state:2}
%   \end{eqnarray}
%   Suppose both \eqref{cascade:state:1} and \eqref{cascade:state:2} has
%   globally asymptotically stable equilibrium points at their respective
%   origins.
% \end{example}

% \begin{lemma}
%   Under the stated assumptions, if the system \eqref{cascade:state:1} with
%   $x_2$ as input, is ISS and the origin of \eqref{cascade:state:2} is globally
%   uniformly asymptotically stable, then the origin of the cascade system
%   \eqref{cascade:state:1} and \eqref{cascade:state:2} is globally uniformly
%   asymptotically stable.
% \end{lemma}