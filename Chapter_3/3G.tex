\section{输入-状态稳定性}\label{3Gref}

考虑系统
\begin{equation} \label{sys:I:ss}
  \dot{x} = f (t, x, u) \text{,}
\end{equation}
其中 $f : (0, \infty) \times \mathbb{R}^n \times \mathbb{R}^n \to \mathbb{R}^n$ 是关于 $t$ 分段连续且关于 $x$ 和 $u$ 局部 Lipschitz 的函数。输入 $u$ 是 $t$ 的有界连续函数。假设无外力系统
\begin{equation}
    \dot{x} = f (t, x, 0)
\end{equation}
具有全局一致渐近稳定的平衡点 $x = 0$。那么,在有界输入 $u(t)$ 下,系统 \eqref{sys:I:ss} 会有怎样的表现?首先我们从简单情形出发,讨论线性时不变系统。

\begin{example}
    对于线性时不变系统
    \[
        \dot{x} = A  x + B  u \text{,}
    \]
    其中 $A$ 是 Hurwitz 的,我们可以得到其解为
    \[
        x (t) = \mathrm{e}^{A (t - t_0)} x (t_0) + \int^t_{t_0} e^{A (t - \tau)} B  u (\tau) \diff \tau \text{。}
    \]
    利用 $\| \mathrm{e}^{A (t - t_0)} \| \leq k  \mathrm{e}^{- \lambda (t - t_0)}$(其中 $k, \lambda \geq 0$),可以对这个解做如下近似:
    \begin{eqnarray*}
        \| x (t) \| & \leq & k  \mathrm{e}^{- \lambda (t - t_0)} \| x (t_0) \| + k
        \int^t_{t_0} \mathrm{e}^{- \lambda (t - \tau)} \| B \| \|  u (\tau) \| \diff \tau\\
        & \leq & k  \mathrm{e}^{- \lambda (t - t_0)} \| x (t_0) \| + k \| B \| \sup_{t_0
        \leq \tau \leq t} \|  u (\tau) \| \int^t_{t_0} \mathrm{e}^{- \lambda (t - \tau)} \diff \tau\\
        & \leq & k  \mathrm{e}^{- \lambda (t - t_0)} \| x (t_0) \| + \frac{k \| B \|}{\lambda} \sup_{t_0 \leq \tau \leq t} \|  u (\tau) \| \text{。}
    \end{eqnarray*}
    由此可知,解当中的零输入响应(上式最后一行的第一项)按指数速率衰减至零,而零状态响应(上式最后一行的第二项)对于每个有界的输入都是有界的。也就是说,对于任意有界的输入,系统的状态都有界。
\end{example}

这样的性质被称为输入-状态稳定性,这里给出其正式定义。

\begin{definition}[输入-状态稳定性(input-to-state stability)] \label{def_iss}
    系统 \eqref{sys:I:ss} 称为 {\textbf{输入-状态稳定(input-to-state stable,ISS)}}\index{输入-状态稳定(input-to-state stable)},如果存在一个 $\mathcal{K}\mathcal{L}$ 类函数 $\beta$ 和一个 $\mathcal{K}$ 类函数 $\gamma$,使得对于任意初始状态 $x (t_0)$ 和任意有界输入 $u (t)$,解 $x (t)$ 对所有 $t \geq t_0$ 均存在且满足
    \[
        \| x (t) \| \leq \beta (\| x (t_0) \|, t - t_0) + \gamma (\sup_{t_0 \leq \tau \leq t} \|  u (\tau) \|) \text{。}
    \]
\end{definition}

\begin{note}
    输入-状态稳定的系统,有如下性质。
    \begin{enumerate}
        \item 对于任意 $u (t)$,$x (t)$ 都会是有界的。
        \item 随着 $t$ 增加,$x (t)$ 是最终有界的,其界是以 $\sup_{t_0 \leq \tau \leq t} \|  u (\tau) \|$ 为自变量的 $\mathcal{K}$ 类函数。
        \item 若 $u (t) \equiv 0$, 则 $x = 0$ 全局一致渐近稳定。
        \item 若 $u (t)$ 随 $t \rightarrow \infty$ 趋于零,则 $x(t)$ 也随 $t \to \infty$ 趋于零。
    \end{enumerate}
\end{note}

接下来,我们讨论非线性系统在什么情况下具有这样的性质。对应的无外力系统具有一致渐近稳定平衡点即可吗?答案是否定的。下面看一个例子。

\begin{example}
    对于系统
    \[
        \dot{x} = -3x + (1 + 2 x^2) u \text{,}
    \]
    当输入 $u = 0$ 时,平衡点 $x = 0$ 是全局渐近稳定的。然而,当 $x(0) = 2$ 且 $u(t) = 1$ 时,可得解为
    \[
        x(t) = \dfrac{3 - \mathrm{e}^t}{3 - 2 \mathrm{e}^t} \text{。}
    \]
    而这个解是无界的,甚至有有限逃逸时间($t = \frac{1}{2} \ln{3}$)。
\end{example}

下面给出输入-状态稳定性的充分条件。

\begin{theorem}[输入-状态稳定性的充分条件]
    令 $V : [ 0, \infty ) \times \mathbb{R}^n \to \mathbb{R}$ 是 $C^1$ 函数,且满足
    \begin{gather*}
        \alpha_1 (\| x \|) \leq V(t, x) \leq \alpha_2 (\| x \|) \\
        \dfrac{\partial V}{\partial t} + \dfrac{\partial V}{\partial x} f(t, x, u) \leq -W_3(x), \quad \forall \| x \| \geq \rho (u) > 0,
    \end{gather*}
    $\forall (t, x, u) \in [ 0, \infty ) \times \mathbb{R}^n \times \mathbb{R}^n,\, \alpha_1, \alpha_2 \in \mathcal{K}_\infty,\, \rho \in \mathcal{K},\, 0 < W_3(x) \in C^0$。则系统 $\dot{x} = f(t, x, u)$ 是 ISS 的,其中定义 \ref{def_iss} 中的 $\gamma = \alpha_1^{-1} \circ \alpha \circ \rho$。
\end{theorem}

\begin{proof}
    讲义上无,待补充。%TODO
\end{proof}

下一条引理将全局指数稳定性和输入-状态稳定性联系了起来。

\begin{lemma}\label{ISS_lemma}
    假设 $f(t, x, u)$ 对 $(x, u)$ 连续可微并且是全局 Lipschitz 的(对 $t$ 一致)。若无外力系统 $\dot{x} = f (t, x, 0)$ 具有全局指数平衡点 $x = 0$,则系统 $\dot{x} = f (t, x, u)$ 是 ISS 的。
\end{lemma}

\begin{proof}
    讲义上无,待补充。%TODO
\end{proof}

% TODO:讲义上的示例

\section*{级联系统}

考虑级联系统
\begin{align*}
    \dot{x}_1 & = f_1 (t, x_1, x_2)  \label{cascade:state:1}\\
    \dot{x}_2 & = f_2 (t, x_2)  \label{cascade:state:2}
\end{align*}
其中子系统 $\dot{x}_1 = f_1 (t, x_1, 0)$ 的平衡点 $x_1 = 0$ 是全局一致渐近稳定的(GUAS),$\dot{x}_2 = f_2(t, x_2)$ 的平衡点 $x_2 = 0$ 也是全局一致渐近稳定的(GUAS)。那么在什么条件下,这个级联系统的原点 $x = (x_1, x_2) = 0$ 也是全局一致渐近稳定的(GUAS)?

这里我们不加证明地给出一条引理。
\begin{lemma}
    在上述假设下,如果将 $x_2$ 作为输入的系统 \eqref{cascade:state:1} 是输入-状态稳定的(ISS),那么该级联系统的原点是全局一致渐近稳定的(GUAS)。
\end{lemma}
