\section{时变正定函数与非自治系统稳定性定理}\label{3Cref}
\begin{definition}[与时间相关的正定、正半定函数;渐弱,径向无界]
    \begin{itemize}[leftmargin=1em]
      \item 称$V(t,x)$是正半定函数\index{正半定(positive semidefinite)},若$V(t,x)\ge 0$,$V(t,0)=0$。
      \item 称$V(t,x)$是正定函数\index{正定(positive definite)},若$V(t,x)\ge W_1(x)$,其中$W_1(x)$是正定函数。
      \item 称$V(t,x)$是{\bf 渐弱的(decrescent)}\index{渐弱的(decrescent)},若$V(t,x)\le W_2(x)$,其中$W_2(x)$是正定函数。
      \item 称$V(t,x)$是径向无界的(radially unbounded)\index{径向无界(radial unboundness)},若$\|x\|\to\infty$,$V(t,x)\to \infty$。
      \item 称$V(t,x)$是负定/负半定函数\index{负定的(negative definite)}\index{负半定的(negative semidefinite)},若$-V(t,x)$是正定/正半定函数。
    \end{itemize}
\end{definition}
\newpage
\begin{example}[正定与渐弱]
    令$x = [\begin{array}{cc}
    x_1& x_2
  \end{array}]^\mathrm{T}$。
  \begin{itemize}[leftmargin=1em]
    \item $V (x, t) = (1 + t^2) (x^2_1 + x_2^2)$:可取 $W_1 (x) = x^2_1 +
    x^2_2$,则其正定;但其并不渐弱。
    
    \item $V (x, t) = (1 + \sin^2 t) (x^2_1 + x_2^2)$:可取$W_1 (x) = x^2_1 +
    x^2_2, W_2 = 2 (x^2_1 + x^2_2)$,则其正定且渐弱。
    
    \item $V (x, t) = \frac{1 + t^2}{2 + t^2} (x^2_1 + x_2^2)$:注意到因子取值范围$(\frac12,1)$,可取 $W_1 (x)
    = \frac{1}{2} (x^2_1 + x^2_2), W_2 = x^2_1 + x^2_2$,则其正定且渐弱。
  \end{itemize}
\end{example}

\begin{theorem}[非自治系统的Lyapunov稳定性定理]\label{lyapunov_nonauto}\index{Lyapunov稳定性定理!非自治系统的$\sim$}
  令 $x = 0$ 是系统 \eqref{freeofnonauto} 的平衡点,$D \subset
  \mathbf{R}^n$ 是包含原点的域。令 $V: D \times [0, \infty) \rightarrow \mathbf{R}$
  是连续可微函数,如果其满足
  \begin{itemize}[leftmargin=1em]
    \item $V (x, t)$ 正定(即存在$W_1 (x) \leq V (x, t)$,其中$W_1$正定)与
    $\dot{V} (x, t) = \frac{\partial V}{\partial x} f (x, t) + \frac{\partial V}{\partial t} \leq 0$(负半定),那么$x=0$是稳定的。
    
    \item 进一步地,如果 $V (x, t)$ 也是渐弱的,也即
    \[ W_1 (x) \leq V (x, t) \leq W_2 (x) \qquad (W_1, W_2 \ \text{是正定的}) \]
    那么 $x = 0$ 是一致稳定的。
    
    \item 进一步地,如果第一点中的第二个条件加强为
    \[ \dot{V} (x, t) = \frac{\partial V}{\partial x} f (x, t) + \frac{\partial
       V}{\partial t} \leq - W_3 (x) \qquad (W_3 \ \text{是正定的}) \]
    那么 $x = 0$ 是一致渐近稳定的。
    
    \item 进一步地,如果$D=\mathbf{R}^n$ 且 $W_1 (x)$ 是径向无界的,那么 $x = 0$ 是全局一致渐近稳定的。

    \item 若前三点中的$W_i$都具有$k_i \| x \|^\alpha $的形式(各$k_i$及$\alpha> 0,i=1,2,3$),那么$x = 0$ 是一致指数稳定的。
  \end{itemize}
\end{theorem}
\begin{note}
  $V(t,x)\le W_2(x)$对于一致稳定性是必要的。
\end{note}
\begin{example}
  考虑下述系统
  \begin{align*}
    \dot{x}_1 & = - x_1 - \mathrm{e}^{- 2 t} x_2\\
    \dot{x}_2 & = x_1 - x_2
  \end{align*}
  考虑下述候选Lyapunov函数
  \[ V (x, t) = x^2_1 + (1 + \mathrm{e}^{- 2 t}) x^2_2 \]
  我们可见
  \[ W_1 (x)\triangleq \| x \|^2 = x^2_1 + x^2_2 \leq V (x, t) \leq x^2_1 + 2
     x^2_2  \leq 2 \| x \|^2 \triangleq W_2 (x) \]
  $W_1$和$W_2$均为正定,$W_1$径向无界。因此,$V (x, t)$是正定、渐弱、径向无界的。
  
  求$V$随时间(沿状态轨线)的导数得
  \begin{align*}
    \dot{V} (x, t) & =  \underbrace{2 x_1 \dot{ x}_1 + (1 + \mathrm{e}^{- 2 t}) 2 x_2 \dot{x}_2}_{\frac{\partial V}{\partial x} f (x, t)}  \underbrace{-2 \mathrm{e}^{- 2 t} x^2_2}_{\frac{\partial V}{\partial t}}\\
    &=-2\mathrm{e}^{-2t}x_{2}^{2}-2x_{1}^{2}-2\mathrm{e}^{-2t}x_{1}x_{2}+2(1+\mathrm{e}^{-2t})(x_{2}x_{1}-x_{2}^{2})\\
    &=-2x_{1}^{2}+2x_{1}x_{2}-2x_{2}^{2}-2\mathrm{e}^{-2t}(x_{2}^{2}+x_{1}x_{2}-x_{2}x_{1}+x_{2}^{2})\\
    & =  - (x^2_1 + x^2_2) - (x_1 - x_2)^2 - 4 \mathrm{e}^{- 2 t} x^2_2\\
    & \leq  - (x^2_1 + x^2_2) = - \| x \|^2\triangleq  - W_3 (x)
  \end{align*}
  因此,原点是全局一致渐近稳定的(由$W_1,W_2,W_3$的形式,可进一步判别指数稳定性)。
\end{example}