\section{Barbalat引理及其推论}\label{3Eref}
类似 \ref{2Dref} 节开头的动机,我们欲在针对非自治系统构造的Lyapunov函数导数仅为负定时,仍可判断系统的渐近稳定性。
遗憾的是,适用于自治系统的LaSalle不变集原理并不适用于非自治系统。

下面要介绍的Barbalat引理(\ref{barbalat})给出了一类本身及导数渐近收敛的函数的刻画。
先看一个“否定直觉”的例子。
\begin{example}
    给定对$t$可微的函数$f$,$f$收敛  $\nLeftrightarrow $ $\dot{f} \rightarrow 0$!
  \begin{itemize}[leftmargin=1em]
    \item $\nRightarrow $: $f = \mathrm{e}^{- t} \sin (\mathrm{e}^{2 t}) \rightarrow 0$,但是$ \dot{f} = - \mathrm{e}^{- t} \sin
    (\mathrm{e}^{2 t}) + e^t \cos (\mathrm{e}^{2 t}) \cdot 2 \mathrm{e}^{2 t} \rightarrow \infty$;
    
    \item $\nLeftarrow $:$f = \ln  t$,有$ \dot{f} = \frac{1}{t} \rightarrow0$,但是$f \rightarrow \infty$。
  \end{itemize}
\end{example}
\begin{definition}[一致连续(uniformly continuous)]
    称函数$f : \mathbf{R} \rightarrow \mathbf{R}$ 是{\bf 一致连续的(uniformly continuous)}\index{一致连续的(uniformly continuous)},若
     \[\forall \varepsilon > 0, \exists \delta
  = \delta (\varepsilon) > 0, \forall | t_2 - t_1 | \leq \delta \Rightarrow |
  f (t_2) - f (t_1) | \leq \varepsilon\]
\end{definition}
\begin{lemma}
    有界闭集(紧集)上的连续函数必是一致连续的。
\end{lemma}
\begin{lemma}
    可微函数一致连续的一个充分条件是其导数有界。
\end{lemma}
\begin{theorem}[Barbalat引理]\label{barbalat}\index{Barbalat引理}
    若可微函数$f(t)$当$t\to\infty$时具有有限极限,且$\dot{f}(t)$是一致连续的,那么$\lim\limits_{t\to\infty}\dot{f}(t)=0$。
\end{theorem}
\begin{proof}
    
\end{proof}
\begin{theorem}[``Lyapunov-like'']
    
\end{theorem}
\begin{proof}
    
\end{proof}
\begin{theorem}[LaSalle-Yoshizawa定理]\index{LaSalle-Yoshizawa定理}\label{LaSalle-Yoshizawa}
    令$x=0$是 \eqref{freeofnonauto} 的平衡点,并设$f$对$t$是分段连续,且对$x$是局部Lipschitz的(对于$t$是一致的)。令$V$是以连续可微函数,满足$\forall t\ge 0,x\in\R{}^n$,\[\gamma_1(\|x\|)\le V(x,t)\le \gamma_2(\|x\|)\]
    \[\begin{aligned}\dot{V}=\frac{\partial V}{\partial t}+\frac{\partial V}{\partial x}f(x,t)\leq-\omega(x)\leq0\end{aligned}\]
    其中$\gamma_1$和$\gamma_2$都是$\mathcal{K}_\infty$类函数且$\omega$为连续函数。那么  \eqref{freeofnonauto} 的所有解均满足$\lim\limits_{t\to\infty}\omega(x(t))=0$。
    
    进一步地,若$\omega(x)$是正定的,那么$x=0$是全局一致渐近稳定的。
\end{theorem}
\begin{proof}
    
\end{proof}