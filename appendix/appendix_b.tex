\chapter{多输入多输出系统的鲁棒自适应控制} \label{MIMO_robust}

% 基于学长提供的笔记整理得到

考虑系统
\begin{equation} \label{MIMO_sys_disturbance}
    \dot{x} = A x + B \Lambda (u + \Theta \Phi(x)) + \xi(t) \text{,}
\end{equation}
其中状态 $x \in \mathbb{R}^n$,控制输入 $u \in \mathbb{R}^m$,矩阵 $B \in \mathbb{R}^{n \times m}$ 已知,矩阵 $A \in \mathbb{R}^{n \times n}$ 和 $\Lambda \in \mathbb{R}^{m \times m}$ 未知,$\Lambda$ 是对角正定阵,矩阵对 $(A, B \Lambda)$ 能控,$\Phi(x) \in \mathbb{R}^n$ 是连续且已知的函数,未知常值参数矩阵 $\Theta \in \mathbb{R}^{n \times m}$,$\xi(t) \in \mathbb{R}^n$ 是时变而一致有界的扰动($\| \xi(t) \| \leq d_{\mathrm{max}}$)。% theta 转置
可采用如下自适应控制律:
\begin{align*}
    u &= \hat{K}(t)x - \hat{\Theta}\Phi(x), \\
    \dot{\hat{K}} &= - \Gamma_1 (B^\mathrm{T} P x x^\mathrm{T} + \Sigma_1\hat{K}), \\
    \dot{\hat{\Theta}} &= \Gamma_2(B^\mathrm{T} P x \Phi(x)^\mathrm{T} - \Sigma_2 \hat{\Theta}) \text{,}
\end{align*}
其中,$\Gamma_1, \Gamma_2, \Sigma_1, \Sigma_2$ 均为对角正定阵。

\begin{proof}
    考虑理想系统——即 \eqref{MIMO_sys_disturbance} 中各参数均已知且 $\xi(t) \equiv 0$,则可设计理想控制律
    \[
        u = K^*x - \Theta \Phi(x) \text{,}
    \]
    其中 $K^* \in \mathbb{R}^{m \times n }$。从而,闭环系统可写为
    \[
        \dot{x} = (A  + B \Lambda K^*) x \text{。}
    \]
    记
    \[
        A  + B \Lambda K^* \triangleq A^* \text{,}
    \]
    系统稳定条件为 $A^*$ 是 Hurwitz 的。由于 $(A,B\Lambda)$ 能控,我们总可以选取到合适的 $K^*$ 实现系统的稳定。

    对于非理想系统 \eqref{MIMO_sys_disturbance},我们设计控制律
    \[
        u = \hat{K}(t) x - \hat{\Theta}(t) \Phi(x) \text{,}
    \]
    其中 $\hat{K}$ 和 $\hat{\Theta}$ 是对参数 $K^*$ 和 $\Theta$ 的估计。记
    \[
        \tilde{K}(t)\triangleq\hat{K}(t)-K^*,\tilde{\Theta}(t)\triangleq\hat{\Theta}(t)-\Theta \text{,}
    \]
    则闭环系统可写作
    \begin{align*}
         \dot{x}=&Ax+B\Lambda (\hat K(t)x-\hat \Theta(t)\Phi(x)+\Theta\Phi(x))+\xi(t)\\
         =&(A+B\Lambda\hat K(t))x-B\Lambda \tilde \Theta(t)\Phi(x)+\xi(t)\\
         =&A^*x+B\Lambda\tilde{K}(t)x-B\Lambda \tilde \Theta(t)\Phi(x)+\xi(t)
     \end{align*}

     考虑如下候选 Lyapunov 函数:
     \[
        V = x^\mathrm{T} P x + \tr (\tilde{K}^\mathrm{T} \Lambda \Gamma_1^{-1}  \tilde{K}) + \tr (\tilde{\Theta}^\mathrm{T} \Lambda\Gamma_2^{-1} \tilde{\Theta}), \quad \Gamma_1 = \Gamma_1^\mathrm{T} > 0, \Gamma_2 = \Gamma_2^\mathrm{T} > 0 \text{,}
     \]
     其中 $P = P^\mathrm{T} > 0$ 对于某个对称正定阵$Q$满足 Lyapunov 方程
     \[
        P A^* + {A^*}^\mathrm{T} P = -Q \text{。}
     \]

     该候选 Lyapunov 函数沿闭环系统轨线的导数为:
     \begin{align*}
         \dot{V}=&\dot{x}^\mathrm{T} P x+ x^\mathrm{T} P \dot{x}+ 2 \tr (  \tilde{K}^\mathrm{T}\Lambda \Gamma_1^{-1}  \dot{\hat{K}})+2 \tr (  \tilde{\Theta}^\mathrm{T} \Lambda\Gamma_2^{-1}  \dot{\hat{\Theta}})\\
        =&(A^*x+B\Lambda\tilde{K}x-B\Lambda \tilde \Theta(t)\Phi(x)+\xi(t))^\mathrm{T}Px\\
        &+x^\mathrm{T}P(A^*x+B\Lambda\tilde{K}x-B\Lambda \tilde \Theta(t)\Phi(x)+\xi(t))\\
        &+2 \tr (  \tilde{K}^\mathrm{T} \Lambda\Gamma_1^{-1}  \dot{\hat{K}})+2 \tr (  \tilde{\Theta}^\mathrm{T} \Lambda\Gamma_2^{-1}  \dot{\hat{\Theta}})\\
        =&x^\mathrm{T} {A^*}^\mathrm{T}Px+x^\mathrm{T}PA^*x
        +x^\mathrm{T}\tilde{K}^\mathrm{T}\Lambda^\mathrm{T}B^\mathrm{T}Px+x^\mathrm{T}PB\Lambda\tilde{K}x\\
        &-\Phi(x)^\mathrm{T}\tilde\Theta(t)^\mathrm{T}\Lambda^\mathrm{T}B^\mathrm{T}Px-x^\mathrm{T}PB\Lambda \tilde \Theta(t)\Phi(x)+\xi(t)^\mathrm{T}Px+x^\mathrm{T}P\xi(t)\\
        &+2 \tr (  \tilde{K}^\mathrm{T} \Lambda\Gamma_1^{-1}  \dot{\hat{K}})+2 \tr (  \tilde{\Theta}^\mathrm{T} \Lambda\Gamma_2^{-1}  \dot{\hat{\Theta}}) \text{。}
     \end{align*}

     我们有
     \begin{gather*}
         x^\mathrm{T} {A^*}^\mathrm{T}Px+x^\mathrm{T}PA^*x=-x^\mathrm{T}Qx \\
         x^\mathrm{T}\tilde{K}^\mathrm{T}\Lambda^\mathrm{T}B^\mathrm{T}Px=x^\mathrm{T}PB\Lambda\tilde{K}x = \tr (\tilde{K}^\mathrm{T}\Lambda B^\mathrm{T}Pxx^\mathrm{T}) \\
         \Phi(x)^\mathrm{T}\tilde\Theta(t)^\mathrm{T}\Lambda^\mathrm{T}B^\mathrm{T}Px=x^\mathrm{T}PB\Lambda \tilde \Theta(t)\Phi(x) =\tr (\tilde{\Theta}^\mathrm{T}\Lambda B^\mathrm{T}Px\Phi(x)^\mathrm{T})
     \end{gather*}

     因此,
     \begin{align*}
        \dot{V}=&-x^\mathrm{T}Qx+2\text{tr}(\tilde{K}^\mathrm{T}\Lambda B^\mathrm{T}Pxx^\mathrm{T})-2\tr(\tilde{\Theta}^\mathrm{T}\Lambda B^\mathrm{T}Px\Phi(x)^\mathrm{T})
        \\
        &+2\tr (  \tilde{K}^\mathrm{T} \Lambda\Gamma_1^{-1}  \dot{\hat{K}})+2\tr (  \tilde{\Theta}^\mathrm{T} \Lambda\Gamma_2^{-1}  \dot{\hat{\Theta}})+2x^\mathrm{T}P\xi(t)\\
        =&-x^\mathrm{T}Qx+2\tr(\tilde{K}^\mathrm{T}\Lambda( B^\mathrm{T}Pxx^\mathrm{T}+\Gamma_1^{-1}  \dot{\hat{K}}))\\&+2\tr(\tilde{\Theta}^\mathrm{T}\Lambda(- B^\mathrm{T}Px\Phi(x)^\mathrm{T}+\Gamma_2^{-1}  \dot{\hat{\Theta}}))+2x^\mathrm{T}P\xi(t)
    \end{align*}

    此处我们考虑 $\sigma$-修正控制律:
    \begin{align*}
        \dot{\hat{K}}&=-\Gamma_1(B^\mathrm{T}Pxx^\mathrm{T}+\Sigma_1\hat{K}), \\ \dot{\hat{\Theta}}&=\Gamma_2(B^\mathrm{T}Px\Phi(x)^\mathrm{T}-\Sigma_2\hat{\Theta})
    \end{align*}
    其中 $\Sigma_1, \Sigma_2$ 是对角正定阵。

    于是我们有:
    \begin{align*}
        \dot{V}=&-x^\mathrm{T}Qx-2\tr(\tilde{K}^\mathrm{T}\Lambda\Sigma_1\hat{K})-2\tr(\tilde{\Theta}^\mathrm{T}\Lambda\Sigma_2\hat{\Theta})+2x^\mathrm{T}P\xi(t)\\
        =&-x^\mathrm{T}Qx-2\tr(\tilde{K}^\mathrm{T}\Lambda\Sigma_1\tilde{K})-2\tr(\tilde{K}^\mathrm{T}\Lambda\Sigma_1 K)-2\tr(\tilde{\Theta}^\mathrm{T}\Lambda\Sigma_2\tilde{\Theta})\\
        &-2\tr(\tilde{\Theta}^\mathrm{T}\Lambda\Sigma_2\Theta)+2x^\mathrm{T}P\xi(t)
    \end{align*}
    根据定义,有:
    \begin{align*}
        \tr(\tilde{K}^\mathrm{T}\Lambda\Sigma_1\tilde{K})&\geq \|\tilde{K}\|_F^2\Lambda_{\mathrm{min}}{\Sigma_1}_{\mathrm{min}} \\
        \tr(\tilde{\Theta}^\mathrm{T}\Lambda\Sigma_2\tilde{\Theta})&\geq \|\tilde{\Theta}\|_F^2\Lambda_{\mathrm{min}}{\Sigma_2}_{\mathrm{min}}
    \end{align*}
    其中 $\Lambda_{\mathrm{min}}, {\Sigma_1}_{\mathrm{min}}, {\Sigma_2}_{\mathrm{min}}$ 分别为 $\Lambda, \Sigma_1, \Sigma_2$ 的最小对角元素。由 Schwartz 不等式,可得
    \begin{align*}
        |\tr(\tilde{K}^\mathrm{T}\Lambda\Sigma_1 K)| &\leq \dfrac{1}{2} \left( \tr(\tilde{K}^\mathrm{T} \Lambda\Sigma_1 \tilde{K}) + \tr(K^\mathrm{T} \Lambda\Sigma_1 K) \right) \\
        |\tr(\tilde{\Theta}^\mathrm{T}\Lambda\Sigma_2 \Theta)| &\leq \dfrac{1}{2} \left( \tr(\tilde{\Theta}^\mathrm{T} \Lambda\Sigma_2 \tilde{\Theta}) + \tr(\Theta^\mathrm{T} \Lambda\Sigma_2 \Theta) \right)
    \end{align*}

    从而,我们有
    \begin{align*}
        \dot{V} \leq & -\lambda_{\min}(Q)\|x\|^2
        -\tr(\tilde{K}^\mathrm{T}\Lambda\Sigma_1\tilde{K})
        -\tr(\tilde{\Theta}^\mathrm{T}\Lambda\Sigma_2\tilde{\Theta})\\
        &+\tr(K^\mathrm{T} \Lambda\Sigma_1 K)+\tr(\Theta^\mathrm{T} \Lambda\Sigma_2 \Theta)+2\|x\|\lambda_{\max}(P)\xi_{\max}\\
        \leq & -\lambda_{\min}(Q)\|x\|^2 - \|\tilde{K}\|_F^2\Lambda_{\mathrm{min}}{\Sigma_1}_{\mathrm{min}} - \|\tilde{\Theta}\|_F^2\Lambda_{\mathrm{min}}{\Sigma_2}_{\mathrm{min}} \\
        &+\tr(K^\mathrm{T} \Lambda\Sigma_1 K)+\tr(\Theta^\mathrm{T} \Lambda\Sigma_2 \Theta)+2\|x\|\lambda_{\max}(P)\xi_{\max}
    \end{align*}

    % 待详细补充 \dot{V} \leq -\beta V + c
    不难看出,$\dot{V}$ 的上界是以 $\| x \|$ 为自变量且具有极大值的二次函数。于是,总存在 $c$ 使得当 $\| x \| > c$ 时 $\dot{V} < 0$。从而该系统是一致最终有界的(UUB)。
\end{proof}


% So we can have,
% \begin{align*}
%     \dot{V}\leq&-\lambda_{min}(Q)\|x\|^2-2\|\tilde{K}\|_F^2\Lambda_{min}{\Sigma_1}_{min}+2\|\tilde{K}\|_F\|K\|_F\|\Lambda\|_F\|\Sigma_1\|_F\\
%     &-2\|\tilde{\Theta}\|_F^2\Lambda_{min}{\Sigma_2}_{min}+2\|\tilde{\Theta}\|_F\|\Theta\|_F\|\Lambda\|_F\|\Sigma_2\|_F+2\|x\|\lambda_{max}(P)\xi_{max}
% \end{align*}
% by root mean square-geometric mean inequality:
%         $2ab \leq a^2 + b^2$,
% $$2\|\tilde{K}\|_F\|K\|_F\|\Lambda\|_F\|\Sigma_1\|_F\leq(\|\tilde{K}\|_F^2+\|K\|_F^2)\|\Lambda\|_F\|\Sigma_1\|_F$$
% $$2\|\tilde{\Theta}\|_F\|\Theta\|_F\|\Lambda\|_F\|\Sigma_2\|_F\leq(\|\tilde{\Theta}\|_F^2+\|\Theta\|_F^2)\|\Lambda\|_F\|\Sigma_2\|_F$$

% So, 
% \begin{align*}
%     \dot{V}\leq&-\lambda_{min}(Q)\|x\|^2-2\|\tilde{K}\|_F^2\Lambda_{min}{\Sigma_1}_{min}+2(\|\tilde{K}\|_F^2+\|K\|_F^2)\|\Lambda\|_F\|\Sigma_1\|_F\\
%     &-2\|\tilde{\Theta}\|_F^2\Lambda_{min}{\Sigma_2}_{min}+2(\|\tilde{\Theta}\|_F^2+\|\Theta\|_F^2)\|\Lambda\|_F\|\Sigma_2\|_F+2\|x\|\lambda_{max}(P)\xi_{max}\\
%     =&-\lambda_{min}(Q)\|x\|^2+2\lambda_{max}(P)\xi_{max}\|x\|-2(\Lambda_{min}{\Sigma_1}_{min}+\|\Lambda\|_F\|\Sigma_1\|_F)\|\tilde{K}\|_F^2\\
%     &-2(\Lambda_{min}{\Sigma_2}_{min}+\|\Lambda\|_F\|\Sigma_2\|_F)\|\tilde{\Theta}\|_F^2+2\|K\|_F^2\|\Lambda\|_F\|\Sigma_1\|_F\\
%     &+2\|\Theta\|_F^2\|\Lambda\|_F\|\Sigma_2\|_F
% \end{align*}

